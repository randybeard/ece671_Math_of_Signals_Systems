%%
% Please see https://bitbucket.org/rivanvx/beamer/wiki/Home for obtaining beamer.
%%
\documentclass{beamer}
\newcommand{\rwbcomment}[1]{{\color{blue}RWB:#1}}
\newcommand{\defeq}{\stackrel{\triangle}{=}}

\title{ECEn 671: Mathematics of Signals and Systems}
\author{Randal W. Beard}
\institute{Brigham Young University}
\date{\today}

\begin{document}

%-------------------------------
\frame{\titlepage}

%-------------------------------
\begin{frame}
\frametitle{Spline problems:}	
\begin{columns}
\begin{column}{0.8\textwidth}
   % lecture material goes here
   \begin{itemize}
   \item Smoothing:  Given a set of data $\{(t_i, y_i)\}$, fit a 5th order polynomial that minimizes the sum of the squared error.
   \item Polynomial fit:  Given a set of data $\{(t_i, y_i)\}$ find the lowest order polynomial that fits the data exactly.
   \item Path planning:  Given a start position and velocity, and an end position and velocity, find a trajectory that satisfies the end points.
   \end{itemize}
\end{column}
\begin{column}{0.2\textwidth}  
	% leave this blank for lecture notes
	\rwbcomment{Add figures}
\end{column}
\end{columns}
\end{frame}

%-------------------------------
\begin{frame}
\frametitle{Path Planning}	
\begin{columns}
\begin{column}{0.8\textwidth}
   % lecture material goes here
   Given start position $p_s\in\mathbb{R}^2$, \\
    start velocity $p_s^{'} \in\mathbb{R}^2$, 
    end position $p_e\in\mathbb{R}^2$, and 
    end velocity $p_e^{'}\in\mathbb{R}^2$.
   
   Suppose that the trajectory will be given by a 4th order spline:
   \[
   p(t) = \begin{pmatrix} c_{0x} \\ c_{0y} \end{pmatrix} + \begin{pmatrix} c_{1x} \\ c_{1y} \end{pmatrix} t + \begin{pmatrix} c_{2x} \\ c_{2y} \end{pmatrix} \frac{t^2}{2!} + \begin{pmatrix} c_{3x} \\ c_{3y} \end{pmatrix} \frac{t^3}{3!}.
   \]
   Rewrite in matrix notation as:
   \begin{align*}
   p(t) &= \begin{pmatrix} c_{0x} & c_{1x} & c_{2x} & c_{3x} \\ c_{0y} & c_{1y} & c_{2y} & c_{3y} \end{pmatrix} \begin{pmatrix} 1 \\ t \\  \frac{t^2}{2!} \\ \frac{t^3}{3!} \end{pmatrix} \\
   	    &= C \phi(t),
   \end{align*}
   where
   \begin{align*}
   C &\defeq	\begin{pmatrix} c_{0x} & c_{1x} & c_{2x} & c_{3x} \\ c_{0y} & c_{1y} & c_{2y} & c_{3y} \end{pmatrix}  \\
   \phi(t) &\defeq \begin{pmatrix} 1 \\ t \\  \frac{t^2}{2!} \\ \frac{t^3}{3!} \end{pmatrix}.
   \end{align*}

\end{column}
\begin{column}{0.2\textwidth}  
	% leave this blank for lecture notes
\end{column}
\end{columns}
\end{frame}

%-------------------------------
\begin{frame}
\frametitle{Path Planning (cont.)}	
\begin{columns}
\begin{column}{0.8\textwidth}
	Then 
	\begin{align*}
	p_s &= p(0) = C \phi(0) \\
	p_s^{'} &= p^{'}(0) = C \phi^{'}(0) \\
	p_e &= p(1) = C \phi(1) \\
	p_e^{'} &= p^{'}(1) = C \phi^{'}(1)
	\end{align*}
	where
	\[
	   \phi^{'}(t) = \frac{d}{dt}\phi(t) = \begin{pmatrix} 0 \\ 1 \\  t \\ \frac{t^2}{2!} \end{pmatrix}.
	\]
	
\end{column}
\begin{column}{0.2\textwidth}  
	% leave this blank for lecture notes
\end{column}
\end{columns}
\end{frame}

%-------------------------------
\begin{frame}
\frametitle{Path Planning (cont.)}	
\begin{columns}
\begin{column}{0.8\textwidth}
	Collecting all of these equations into a matrix we get
	\begin{align*}
	\begin{pmatrix} p_s & p_s^{'} & p_e & p_e^{'} \end{pmatrix} &= \begin{pmatrix} C \phi(0) & C \phi^{'}(0) & C \phi(1) & C \phi^{'}(1) \end{pmatrix} \\
	&= C \begin{pmatrix} \phi(0) & \phi^{'}(0) & \phi(1) & \phi^{'}(1) \end{pmatrix}
	\end{align*}
	or
	\[
	P = C \Phi
	\]
	where
	\begin{align*}
	P &= \begin{pmatrix} p_s & p_s^{'} & p_e & p_e^{'} \end{pmatrix} \\
	\Phi &= \begin{pmatrix} \phi(0) & \phi^{'}(0) & \phi(1) & \phi^{'}(1) \end{pmatrix}.
	\end{align*}
	
	We can find $C$ by inverting $\Phi$ as
	\[
	C = P \Phi^{-1}
	\]
	
\end{column}
\begin{column}{0.2\textwidth}  
	% leave this blank for lecture notes
\end{column}
\end{columns}
\end{frame}


%-------------------------------
\begin{frame}
\frametitle{Path Planning (cont.)}	
\begin{columns}
\begin{column}{0.8\textwidth}
	But what if we use a 4th order polynomial:
	\begin{align*}
   	p(t) &= \begin{pmatrix} c_{0x} \\ c_{0y} \end{pmatrix} + \begin{pmatrix} c_{1x} \\ c_{1y} \end{pmatrix} t + \begin{pmatrix} c_{2x} \\ c_{2y} \end{pmatrix} \frac{t^2}{2!} + \begin{pmatrix} c_{3x} \\ c_{3y} \end{pmatrix} \frac{t^3}{3!} + \begin{pmatrix} c_{4x} \\ c_{4y} \end{pmatrix} \frac{t^4}{4!} \\
   	   &= C \phi(t)
   \end{align*}
      where
   \begin{align*}
   C &\defeq	\begin{pmatrix} c_{0x} & c_{1x} & c_{2x} & c_{3x} & c_{4x} \\ c_{0y} & c_{1y} & c_{2y} & c_{3y} & c_{4y} \end{pmatrix}  \\
   \phi(t) &\defeq \begin{pmatrix} 1 \\ t \\  \frac{t^2}{2!} \\ \frac{t^3}{3!} \\ \frac{t^4}{4!} \end{pmatrix}.
   \end{align*}
\end{column}
\begin{column}{0.2\textwidth}  
	% leave this blank for lecture notes
\end{column}
\end{columns}
\end{frame}

%-------------------------------
\begin{frame}\frametitle{Path Planning (cont.)}	
\begin{columns}
\begin{column}{0.8\textwidth}
	Proceeding as before, and collecting into a matrix we get
	\begin{align*}
	\begin{pmatrix} p_s & p_s^{'} & p_e & p_e^{'} \end{pmatrix} &= \begin{pmatrix} C \phi(0) & C \phi^{'}(0) & C \phi(1) & C \phi^{'}(1) \end{pmatrix} \\
	&= C \begin{pmatrix} \phi(0) & \phi^{'}(0) & \phi(1) & \phi^{'}(1) \end{pmatrix}
	\end{align*}
	or
	\[
	P = C \Phi
	\]
	but in this case $P$ is $2\times 4$, $C$ is $2\times 5$ and $\Phi$ is $5\times 4$, and so we can't simply invert $\Phi$, because an inverse does not exist.
	
	How should we proceed?
\end{column}
\begin{column}{0.2\textwidth}  
	% leave this blank for lecture notes
\end{column}
\end{columns}
\end{frame}




\end{document}
