\documentclass{beamer}
\usepackage{amsmath,amsbsy,amsopn,amstext,amsfonts,amssymb}
\usepackage{isomath}
\usepackage{ulem}
%\linespread{1.6}  % double spaces lines
\usepackage{graphicx}
\usepackage{subfigure}
\usepackage{color}
\usepackage{optidef}  % define optimization problems
\usepackage{multicol}  % multiple columns
\usepackage{listings} % for python code
\usepackage{mathrsfs}

\usepackage{polynom}
\newcommand{\adj}{\mathrm{adj}}
\newcommand{\constrainedmin}[3]{
		\begin{mini*}|s|
		{#2}{#1}{}{}
		\addConstraint{#3}
		\end{mini*}
}

\newcommand{\rwbcomment}[1]{{\color{blue}RWB:#1}}
\newcommand{\defeq}{\stackrel{\triangle}{=}}
\newcommand{\abs}[1]{\left|#1\right|}
\newcommand{\norm}[1]{\left\|#1\right\|}
\newcommand{\iprod}[1]{\left<#1\right>}
\newcommand{\ellbf}{\boldsymbol{\ell}}
\newcommand{\nubf}{\boldsymbol{\nu}}
\newcommand{\mubf}{\boldsymbol{\mu}}
\newcommand{\abf}{\mathbf{a}}
\newcommand{\bbf}{\mathbf{b}}
\newcommand{\cbf}{\mathbf{c}}
\newcommand{\dbf}{\mathbf{d}}
\newcommand{\ebf}{\mathbf{e}}
\newcommand{\fbf}{\mathbf{f}}
\newcommand{\gbf}{\mathbf{g}}
\newcommand{\hbf}{\mathbf{h}}
\newcommand{\ibf}{\mathbf{i}}
\newcommand{\jbf}{\mathbf{j}}
\newcommand{\kbf}{\mathbf{k}}
\newcommand{\lbf}{\mathbf{l}}
\newcommand{\mbf}{\mathbf{m}}
\newcommand{\nbf}{\mathbf{n}}
\newcommand{\obf}{\mathbf{o}}
\newcommand{\pbf}{\mathbf{p}}
\newcommand{\qbf}{\mathbf{q}}
\newcommand{\rbf}{\mathbf{r}}
\newcommand{\sbf}{\mathbf{s}}
\newcommand{\tbf}{\mathbf{t}}
\newcommand{\ubf}{\mathbf{u}}
\newcommand{\vbf}{\mathbf{v}}
\newcommand{\wbf}{\mathbf{w}}
\newcommand{\xbf}{\mathbf{x}}
\newcommand{\ybf}{\mathbf{y}}
\newcommand{\zbf}{\mathbf{z}}
\newcommand{\Jbf}{\mathbf{J}}
\newcommand{\Acal}{\mathcal{A}}
\newcommand{\Bcal}{\mathcal{B}}
\newcommand{\Lcal}{\mathcal{L}}
\newcommand{\Ncal}{\mathcal{N}}
\newcommand{\Rcal}{\mathcal{R}}
\definecolor{darkolivegreen}{rgb}{0.33, 0.42, 0.18}

\makeatletter
\newenvironment<>{proofstart}[1][\proofname]{%
    \par
    \def\insertproofname{#1\@addpunct{.}}%
    \usebeamertemplate{proof begin}#2}
  {\usebeamertemplate{proof end}}
\newenvironment<>{proofcont}{%
  \setbeamertemplate{proof begin}{\begin{block}{}}
    \par
    \usebeamertemplate{proof begin}}
  {\usebeamertemplate{proof end}}
\newenvironment<>{proofend}{%
    \par
    \pushQED{\qed}
    \setbeamertemplate{proof begin}{\begin{block}{}}
    \usebeamertemplate{proof begin}}
  {\popQED\usebeamertemplate{proof end}}
\makeatother

\title{ECEn 671: Mathematics of Signals and Systems}
\author{Randal W. Beard}
\institute{Brigham Young University}
\date{\today}

\begin{document}

%-------------------------------
\begin{frame}
	\titlepage
\end{frame}

%%%%%%%%%%%%%%%%%%%%%%%%%%%%%%%%%%%%%%%%%%%%%%%%%%%%%%%%%%%%%%%%%%%%%%%
\section{Metric Spaces}
\frame{\sectionpage}

%----------------------------------
\begin{frame}\frametitle{Spaces}

\begin{itemize}
\item One of the objectives of this course is to develop tools that work in a wide variety of settings.  
\item We will mostly focus on \underline{finite dimensional Hilbert spaces}, which include:
	\begin{itemize}
		\item $\mathbb{R}^n$, $\mathbb{C}^n$, $\mathbb{C}^{m\times n}$,
		\item the set of all functions with finite integral,
		\item the set of all finitely summable sequences,
		\item binary vectors, binary sequences.
	\end{itemize}
\item But does not include important objects like 
	\begin{itemize}
	\item rotations matrices, quaternions, homogeneous transformations.	
	\end{itemize}
\item To make things clear, we will develop the theory systematically in the following order:
	\begin{enumerate}
		\item Metric space
		\item Norm space / Banach space
		\item Inner product space / Hilbert space
	\end{enumerate}
\end{itemize}
\end{frame}

%----------------------------------
\begin{frame}\frametitle{Metric Spaces}

\begin{definition}[Metric Space] A \underline{metric space} is a pair $(\mathbb{X},d)$ where $\mathbb{X}$ is a set and 
	\[
	d:\mathbb{X} \times \mathbb{X} \to \mathbb{R}
	\]
	is a metric defined over $\mathbb{X}$.\\
\end{definition}
	
A metric is a measure of distance between elements in a set.
\end{frame}

%----------------------------------
\begin{frame}\frametitle{Metric Spaces}
\begin{definition}[Metric]
Let $\mathbb{X}$ be a set.  Then 
$d:\mathbb{X} \times \mathbb{X} \to \mathbb{R}$ is a metric if:

\begin{align*} 
(M1) \qquad & d(x,y) = d(y,x), \quad \forall x,y \in \mathbb{X} \\
(M2) \qquad & d(x,y) \geq 0, \quad \forall x,y \in \mathbb{X} \\
(M3) \qquad & d(x,y) = 0, \quad \iff x = y\\
(M4) \qquad &d(x,z) \leq d(x,y) + d(y,z), \quad \forall x,y,z \in \mathbb{X}
\end{align*}
\end{definition}

(M4) is called the \underline{Triangle inequality}.
\end{frame}

%----------------------------------
\begin{frame}\frametitle{Examples of Metric Spaces}

\begin{example}[E1]

$(\mathbb{R}, d)$ where $d(x,y) \defeq |x-y|$ is a metric space.

Note that 
\begin{itemize}
	\item (M1) $|x-y| = |y-x|$, $\forall x,y \in \mathbb{R}$.
	\item (M2) $|x-y|\geq 0$, $\forall x,y \in \mathbb{R}$.	
	\item (M3) $|x-y| = 0$, if $x=y$.	
	\item (M4) $|x-z|\leq |x-y| + |y-z|$ $\forall x,y,z \in \mathbb{R}$. 
\end{itemize}
\end{example}

To convince yourself (M4), draw a picture.  Note, a picture is not a proof.

\end{frame}

%----------------------------------
\begin{frame}\frametitle{Examples of Metric Spaces}

\begin{example}[E2]

$(\mathbb{R}^n, d)$ where 
\[
d(x,y) \defeq \left( \sum_{i=1}^{n} |x_i - y_i|^2 \right)^{\frac{1}{2}}
\]
where $x=(x_1, \dots, x_n)^\top$ and $y=(y_1, \dots, y_n)^\top$.

Verify that $d(\cdot, \cdot)$ satisfies (M1)-(M4).
\end{example}
\end{frame}

%----------------------------------
\begin{frame}\frametitle{Examples of Metric Spaces}

\begin{example}[E3]

$(\mathbb{R}^n, d)$ where 
\[
d(x,y) \defeq \left( \sum_{i=1}^{n} |x_i - y_i|^p \right)^{\frac{1}{p}}
\]
where $p\geq 1$.

For general $p\geq 1$, the triangle inequality is a nontrivial and famous results.
\end{example}

\end{frame}

%----------------------------------
\begin{frame}\frametitle{Examples of Metric Spaces}

\begin{example}[E4 bounded sequence space]

Let $\boldsymbol{\ell}^\infty$ be the set of all sequences of complex numbers where each number is bounded, i.e., 
\[
x = (x_1, x_2, x_3, \dots) \in \boldsymbol{\ell}
\]
if $x_i\in\mathbb{C}$ and $\abs{x_i}<\infty$.

$(\boldsymbol{\ell}, d)$ is a metric space where
\[
d(x, y) = \sup_{j\in\mathbb{N}} \abs{x_i - y_i}.
\]
\end{example}

Verify (M1)-(M4).
\end{frame}

%----------------------------------
\begin{frame}\frametitle{Examples of Metric Spaces}

\begin{example}[E5 continuous function space]
\begin{itemize}
\item Let $C[a,b]$ be the set of all continuous functions on $[a,b]$, i.e., \\
\indent \indent i.e. $ x \in C[a,b] \Rightarrow x(t) $ is continuous on $[a,b]$.\\
Let 
\[ 
d(x,y) = \max_{t \in [a,b]} |x(t)-y(t)| 
\]
then $(C[a, b], d)$ is a metric space.

\item This is a different perspective than calculus.  In calculus you consider one function at a time.  In this class, a function is one \underline{point} in a larger metric space.
\end{itemize}
\end{example}
\end{frame}

%----------------------------------
\begin{frame}\frametitle{Examples of Metric Spaces}

\begin{example}[E6 discrete metric space]

Let $\mathbb{X}$ be any set, e.g., the set of three legged dogs, and let
\[
d(x,y) = \begin{cases}
 1 & \text{~if~} x\neq y \\
 0 & \text{~otherwise}	
 \end{cases}.
 \]
 Then $(\mathbb{X}, d)$ is a metric space since
\begin{itemize}
	\item (M1) $d(x,y) = d(y,x)$, $\forall x,y \in \mathbb{X}$.
	\item (M2) $d(x,y) \geq 0$, $\forall x,y \in \mathbb{X}$.	
	\item (M3) $d(x,y) = 0$, if $x=y$.	
	\item (M4) $d(x,z)\leq d(x,y) + d(y,z)$ $\forall x,y,z \in \mathbb{X}$. 
\end{itemize}
\end{example}
\end{frame}

%----------------------------------
\begin{frame}\frametitle{Examples of Metric Spaces}

\begin{example}[E7 binary vector space]

Let $\mathbb{X}=\{0,1\}^n$ be the set of binary vectors, i.e \\
$x\in\mathbb{X} \Rightarrow x=(x_1, x_2, \dots, x_n)$ where $x_i\in \{0, 1\}$.
Let 
\[ 
d(x,y) = \sum_{i=1}^{n}{h(x_i - y_i)} 
\]  
where
\[ 
h(w) = \begin{cases} 
	1 & \text{ if } \qquad w \neq 0 \\ 
	0 & \text{ otherwise }
\end{cases}.
\]
$h$ is called the hamming distance, and simply counts the number of elements in $x$ and $y$ that are different.
\end{example}
\end{frame}

%----------------------------------
\begin{frame}\frametitle{Metric Spaces / Norm Spaces / Inner Product Spaces}

\begin{itemize}
\item Later in the chapter, 	we will later introduce the concepts of a norm and a norm space, and an inner product and inner product spaces.
\item Many of the metric spaces introduced above are also norm spaces and inner product spaces, but not all.  
\item Metric spaces are the most general of the three.  
\item Before introducing the concept of a norm and a normed space, we develop general tools that also work for metric spaces.
\end{itemize}
\end{frame}



\end{document}