\documentclass{beamer}
\usepackage{amsmath,amsbsy,amsopn,amstext,amsfonts,amssymb}
\usepackage{isomath}
\usepackage{ulem}
%\linespread{1.6}  % double spaces lines
\usepackage{graphicx}
\usepackage{subfigure}
\usepackage{color}
\usepackage{optidef}  % define optimization problems
\usepackage{multicol}  % multiple columns
\usepackage{listings} % for python code
\usepackage{mathrsfs}

\usepackage{polynom}
\newcommand{\adj}{\mathrm{adj}}
\newcommand{\constrainedmin}[3]{
		\begin{mini*}|s|
		{#2}{#1}{}{}
		\addConstraint{#3}
		\end{mini*}
}

\newcommand{\rwbcomment}[1]{{\color{blue}RWB:#1}}
\newcommand{\defeq}{\stackrel{\triangle}{=}}
\newcommand{\abs}[1]{\left|#1\right|}
\newcommand{\norm}[1]{\left\|#1\right\|}
\newcommand{\iprod}[1]{\left<#1\right>}
\newcommand{\ellbf}{\boldsymbol{\ell}}
\newcommand{\nubf}{\boldsymbol{\nu}}
\newcommand{\mubf}{\boldsymbol{\mu}}
\newcommand{\abf}{\mathbf{a}}
\newcommand{\bbf}{\mathbf{b}}
\newcommand{\cbf}{\mathbf{c}}
\newcommand{\dbf}{\mathbf{d}}
\newcommand{\ebf}{\mathbf{e}}
\newcommand{\fbf}{\mathbf{f}}
\newcommand{\gbf}{\mathbf{g}}
\newcommand{\hbf}{\mathbf{h}}
\newcommand{\ibf}{\mathbf{i}}
\newcommand{\jbf}{\mathbf{j}}
\newcommand{\kbf}{\mathbf{k}}
\newcommand{\lbf}{\mathbf{l}}
\newcommand{\mbf}{\mathbf{m}}
\newcommand{\nbf}{\mathbf{n}}
\newcommand{\obf}{\mathbf{o}}
\newcommand{\pbf}{\mathbf{p}}
\newcommand{\qbf}{\mathbf{q}}
\newcommand{\rbf}{\mathbf{r}}
\newcommand{\sbf}{\mathbf{s}}
\newcommand{\tbf}{\mathbf{t}}
\newcommand{\ubf}{\mathbf{u}}
\newcommand{\vbf}{\mathbf{v}}
\newcommand{\wbf}{\mathbf{w}}
\newcommand{\xbf}{\mathbf{x}}
\newcommand{\ybf}{\mathbf{y}}
\newcommand{\zbf}{\mathbf{z}}
\newcommand{\Jbf}{\mathbf{J}}
\newcommand{\Acal}{\mathcal{A}}
\newcommand{\Bcal}{\mathcal{B}}
\newcommand{\Lcal}{\mathcal{L}}
\newcommand{\Ncal}{\mathcal{N}}
\newcommand{\Rcal}{\mathcal{R}}
\definecolor{darkolivegreen}{rgb}{0.33, 0.42, 0.18}

\makeatletter
\newenvironment<>{proofstart}[1][\proofname]{%
    \par
    \def\insertproofname{#1\@addpunct{.}}%
    \usebeamertemplate{proof begin}#2}
  {\usebeamertemplate{proof end}}
\newenvironment<>{proofcont}{%
  \setbeamertemplate{proof begin}{\begin{block}{}}
    \par
    \usebeamertemplate{proof begin}}
  {\usebeamertemplate{proof end}}
\newenvironment<>{proofend}{%
    \par
    \pushQED{\qed}
    \setbeamertemplate{proof begin}{\begin{block}{}}
    \usebeamertemplate{proof begin}}
  {\popQED\usebeamertemplate{proof end}}
\makeatother

\title{ECEn 671: Mathematics of Signals and Systems}
\author{Randal W. Beard}
\institute{Brigham Young University}
\date{\today}

\begin{document}

%-------------------------------
\begin{frame}
	\titlepage
\end{frame}



%%%%%%%%%%%%%%%%%%%%%%%%%%%%%%%%%%%%%%%%%%%%%%%%%%%%%%%%%%%%%%%%%%%%%%%
\section{Linear Operators}
\frame{\sectionpage}

%----------------------------------
\begin{frame}\frametitle{Operators and Transformations}
\begin{definition}[Linear Operator]	
	Let $\mathcal{L}:\mathbb{X}\to\mathbb{Y}$ be an operator from $\mathbb{X}$ to
$\mathbb{Y}$.  $\mathcal{L}$ is a \underline{linear operator} if
\begin{enumerate}
  \item $\mathcal{L}[\alpha x] = \alpha \mathcal{L}[x] \qquad \forall x \in \mathbb{X} \qquad \forall \alpha \in \mathbb{F}$
  \item $\mathcal{L}[x_1 + x_2] = \mathcal{L}[x_1] + \mathcal{L}[x_2], \qquad \forall x_1, x_2\in\mathbb{X}$
\end{enumerate}
\end{definition}

\end{frame}

%----------------------------------
\begin{frame}\frametitle{Examples of Linear Operators}
\begin{example}[Matrices]	
Operators from $\mathbb{C}^n$ to $\mathbb{C}^m$ are $m \times n$ matrices.
\[ A(\alpha x + \beta y) = \alpha Ax + \beta Ay \] 
A is a linear operator.
\end{example}

\begin{example}[Differential Equations with no input]
The differential equation $\dot{x} = Ax; \quad x(0) = x_0$
defines a linear operator from $\mathbb{R}^n$ to $L_2[0,T]$
\[
y(t) = \mathcal{L}[x_0] \text{ where } \mathcal{L}[x_0] = e^{At}x_0 
\]
\indent $\mathcal{L}$ is linear since 
\[ 
e^{At}(\alpha x_{01} + \beta  x_{01}) = \alpha e^{At}x_{01} + \beta e^{At}x_{02} 
\]
\end{example}


\end{frame}

%----------------------------------
\begin{frame}\frametitle{Examples of Linear Operators}
\begin{example}[Convolution]
Convolution is a linear operator from $L_{\infty}$ to
$L_{\infty}$ if $h(t) \in L_1[-\infty,\infty]$, i.e.
\[ y(t) = \mathcal{L}[x(t)] = \int_{-\infty}^{\infty} h(\tau)x(t - \tau)
d\tau \] 
\indent (Recall: for a system to be BIBO stable required
that $\int_{-\infty}^{\infty}|h(\tau)|d\tau <\infty$
\indent i.e. $h(t) \in L_1[-\infty,\infty]$)
\end{example}
\end{frame}

%----------------------------------
\begin{frame}\frametitle{Examples of Linear Operators}
\begin{example}[Fourier Transform]
\noindent (E4) The Fourier transform defines a linear operator from $L_2[-\infty,\infty]$ to $L_2[-\infty,\infty]$.\\
\[ X(j\omega) = \mathcal{L}[x(t)] \defeq \int_{-\infty}^{\infty} x(t)e^{-j\omega t}dt \]	
\end{example}

\vspace{1cm}
{\em There are many examples of linear operators!}
\end{frame}

%----------------------------------
\begin{frame}\frametitle{Range and Null Space of an Operator}
\begin{definition}[Range Space]
	Let $\mathcal{L}: \mathbb{X} \to \mathbb{Y}$ be a
linear operator.  The \underline{range space} (or \underline{image}) of $\mathcal{L}$ is 
\[
\mathcal{R}(\mathcal{L}) = \{ y \in \mathbb{Y} : y = \mathcal{L}[x] \text{ and } x \in \mathbb{X}\} \subseteq \mathbb{Y}
\]
\end{definition}
\begin{definition}[Null Space]
The \underline{Null space} or \underline{kernel} of $\mathcal{L}$ is
\[
\mathcal{N}(\mathcal{L}) = \{ x \in \mathbb{X} : \mathcal{L}[x] = 0 \} \subseteq \mathbb{X} 
\]	
\end{definition}
	
\end{frame}

%----------------------------------
\begin{frame}\frametitle{Example of Range and Null Space}
\begin{itemize}
\item Consider the matrix $A=\begin{pmatrix} 1 & 0 & 0\\ 0 & 0 & 0 \end{pmatrix}$ which defines a linear operator from $\mathbb{R}^3$ to $\mathbb{R}^2$.
\item Note that $y=Ax = 	\begin{pmatrix} 1 & 0 & 0 \\ 0 & 0 & 0 \end{pmatrix}\begin{pmatrix} x_1 \\ x_2 \\ x_3 \end{pmatrix} = \begin{pmatrix} x_1 \\ 0 \end{pmatrix}$.
\item Therefore, the range space is
	\[
	\mathcal{R}(A) = \left\{\begin{pmatrix}\alpha \\ 0 \end{pmatrix} : \alpha\in\mathbb{R} \right\} \subset \mathbb{R}^2.
	\]
\item Similarly, the null space is
	\[
	\mathcal{N}(A) = \left\{\begin{pmatrix}0 \\ \alpha \\ \beta \end{pmatrix} : \alpha, \beta\in\mathbb{R} \right\}\subset \mathbb{R}^3.
	\]
\end{itemize}
\end{frame}



\end{document}