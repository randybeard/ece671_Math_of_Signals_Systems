\documentclass{beamer}
\usepackage{amsmath,amsbsy,amsopn,amstext,amsfonts,amssymb}
\usepackage{isomath}
\usepackage{ulem}
%\linespread{1.6}  % double spaces lines
\usepackage{graphicx}
\usepackage{subfigure}
\usepackage{color}
\usepackage{optidef}  % define optimization problems
\usepackage{multicol}  % multiple columns
\usepackage{listings} % for python code
\usepackage{mathrsfs}

\usepackage{polynom}
\newcommand{\adj}{\mathrm{adj}}
\newcommand{\constrainedmin}[3]{
		\begin{mini*}|s|
		{#2}{#1}{}{}
		\addConstraint{#3}
		\end{mini*}
}

\newcommand{\rwbcomment}[1]{{\color{blue}RWB:#1}}
\newcommand{\defeq}{\stackrel{\triangle}{=}}
\newcommand{\abs}[1]{\left|#1\right|}
\newcommand{\norm}[1]{\left\|#1\right\|}
\newcommand{\iprod}[1]{\left<#1\right>}
\newcommand{\ellbf}{\boldsymbol{\ell}}
\newcommand{\nubf}{\boldsymbol{\nu}}
\newcommand{\mubf}{\boldsymbol{\mu}}
\newcommand{\abf}{\mathbf{a}}
\newcommand{\bbf}{\mathbf{b}}
\newcommand{\cbf}{\mathbf{c}}
\newcommand{\dbf}{\mathbf{d}}
\newcommand{\ebf}{\mathbf{e}}
\newcommand{\fbf}{\mathbf{f}}
\newcommand{\gbf}{\mathbf{g}}
\newcommand{\hbf}{\mathbf{h}}
\newcommand{\ibf}{\mathbf{i}}
\newcommand{\jbf}{\mathbf{j}}
\newcommand{\kbf}{\mathbf{k}}
\newcommand{\lbf}{\mathbf{l}}
\newcommand{\mbf}{\mathbf{m}}
\newcommand{\nbf}{\mathbf{n}}
\newcommand{\obf}{\mathbf{o}}
\newcommand{\pbf}{\mathbf{p}}
\newcommand{\qbf}{\mathbf{q}}
\newcommand{\rbf}{\mathbf{r}}
\newcommand{\sbf}{\mathbf{s}}
\newcommand{\tbf}{\mathbf{t}}
\newcommand{\ubf}{\mathbf{u}}
\newcommand{\vbf}{\mathbf{v}}
\newcommand{\wbf}{\mathbf{w}}
\newcommand{\xbf}{\mathbf{x}}
\newcommand{\ybf}{\mathbf{y}}
\newcommand{\zbf}{\mathbf{z}}
\newcommand{\Jbf}{\mathbf{J}}
\newcommand{\Acal}{\mathcal{A}}
\newcommand{\Bcal}{\mathcal{B}}
\newcommand{\Lcal}{\mathcal{L}}
\newcommand{\Ncal}{\mathcal{N}}
\newcommand{\Rcal}{\mathcal{R}}
\definecolor{darkolivegreen}{rgb}{0.33, 0.42, 0.18}

\makeatletter
\newenvironment<>{proofstart}[1][\proofname]{%
    \par
    \def\insertproofname{#1\@addpunct{.}}%
    \usebeamertemplate{proof begin}#2}
  {\usebeamertemplate{proof end}}
\newenvironment<>{proofcont}{%
  \setbeamertemplate{proof begin}{\begin{block}{}}
    \par
    \usebeamertemplate{proof begin}}
  {\usebeamertemplate{proof end}}
\newenvironment<>{proofend}{%
    \par
    \pushQED{\qed}
    \setbeamertemplate{proof begin}{\begin{block}{}}
    \usebeamertemplate{proof begin}}
  {\popQED\usebeamertemplate{proof end}}
\makeatother

\title{ECEn 671: Mathematics of Signals and Systems}
\author{Randal W. Beard}
\institute{Brigham Young University}
\date{\today}

\begin{document}

%-------------------------------
\begin{frame}
	\titlepage
\end{frame}



%%%%%%%%%%%%%%%%%%%%%%%%%%%%%%%%%%%%%%%%%%%%%%%%%%%%%%%%%%%%%%%%%%%%%%%
\section{Notions of Convergence}
\frame{\sectionpage}

%----------------------------------
\begin{frame}\frametitle{Notions of Convergence}

\begin{definition}[Strong Convergence/ Convergence in norm]
$x_n$ converges strongly to $x$, i.e. $x_n \overset{s}{\to} x$ iff
\[
\norm{ x_n-x}  \to 0 \quad \mathit{ as } \quad n \to \infty 
\]	
\end{definition}

\begin{definition}[Weak Convergence / Convergence in inner product]
$x_n$ converges weakly to $x$, i.e. $x_n \overset{w}{\to} x$ iff
\[ 
\iprod{ x_n, y } \to \iprod{ x,y }, \forall y \in S, 
\]	
\end{definition}

Note that this must hold for all $y \in S$, therefore Example 2.4.4 in the book is bogus!

\end{frame}

%----------------------------------
\begin{frame}\frametitle{Notions of Convergence (cont.)}
\begin{theorem}[Strong vs. Weak Convergence]
  Let $(x_n)$ be a sequence in a normed space $\mathbb{X}$.  Then
\begin{description}
  \item[A.] Strong convergence $\Rightarrow$ weak convergence with the same limit
  \item[B.] The converse of (A.) is not generally true
  \item[C.] If dim $\mathbb{X} < \infty$, then weak convergence $\Rightarrow$ strong convergence.
\end{description}
\end{theorem}
	
\end{frame}

%----------------------------------
\begin{frame}\frametitle{Proof:}
\noindent (A) By definition of strong convergence,
\[ 
x_n \overset{s}{\to} x^* \quad \Rightarrow \quad \norm{ x_n-x^*}  \to 0
\]
so let $y$ be \underline{any} element in $\mathbb{X}$ then
\[ 
|\iprod{ x_n,y } - \iprod{ x^*,y } | = |\iprod{ x_n - x^*,y } | \leq \norm{ x_n-x^*}  \norm{ y}  
\]
but the RHS $\to 0$ which implies that the LHS $\to 0$ which implies weak convergence.
\end{frame}

%----------------------------------
\begin{frame}\frametitle{Proof:}
\noindent (B) Before proving part (B) lets first understand what is wrong with Example 2.4.4 in the book.
\begin{align*}
x_n &= (0,0,0,\ldots,1,0,\ldots) \\
y &= (1, 1/2, 1/4, 1/8, \ldots)
\end{align*}
Then $\iprod{ x_n, y } \to 0$ but this does not imply weak convergence since it must hold for all $y \in \mathbb{X}$.
\end{frame}

%----------------------------------
\begin{frame}\frametitle{Proof:}
To prove part (B) we need a counter example.  Again let $x_n = (0,0,\ldots,0,1,0,\ldots)$ and let $\mathbb{X} = \ellbf_2$ i.e. 
\begin{align*}
y \in \mathbb{X} &\Rightarrow \left( \displaystyle \sum_{i=1}^{\infty}|y_i|^2\right)^{\frac{1}{2}} < \infty \\
	&\Rightarrow y_i \to 0 \quad \mathit{as} \quad i \to \infty 
\end{align*}
so 
\[ 
\iprod{ x_n,y } = y_n \to 0 \quad \mathit{as} \quad n \to \infty \qquad \forall y \in \mathbb{X}
\]
\[ 
\Rightarrow \{x_n\} \overset{w}{\to} 0 
\]
but there is no $x^*$ such that $\norm{ x_n - x^*} \to 0$.
\end{frame}

%----------------------------------
\begin{frame}\frametitle{Proof:}
\noindent (C) Suppose that $x_n \overset{w}{\to} x$ and $dim(\mathbb{X})=k$ then 
\[
\forall y \in \mathbb{X} \qquad \iprod{ x_n,y } \to \iprod{ x,y }.
\]
Let $\{e_1,\ldots,e_k\}$ be an orthonormal basis for $\mathbb{X}$, i.e. $ \iprod{ e_i, e_j } = \delta_{ij}$, then 
\begin{align*}
x_n &= a_1^{(n)}e_1 + \cdots + a_k^{(n)}e_k\\
x &= a_1e_1 + \cdots + a_ke_k.
\end{align*}
\end{frame}

%----------------------------------
\begin{frame}\frametitle{Proof:}

Then since $\iprod{ x_n,y } \to \iprod{ x,y }$ $\forall y$, let $y = e_j$
\[ \Rightarrow \iprod{ a_1^{(n)}e_1+\cdots+a_k^ne_k,e_j } = a_j^{(n)} \]
and \[ \iprod{ a_1e_1 + \cdots + a_ke_k,e_j } = a_j \]
so \[ \iprod{ x_n,e_j } \to \iprod{ x,e_j } \Rightarrow a_j^{(n)} \to a_j \qquad \forall y = 1,\ldots,k\]
Also,
\begin{flalign*}
\norm{ x_n - x}  &= \norm{  \sum_{j=1}^{k} a_j^{(n)}e_j - \sum_{j=1}^{k}a_je_j }  = \norm{  \sum_{j=1}^{k}(a_j^{(n)}-a_j)e_j } \\
            &\leq \sum_{j=1}^{k} | a_j^{(n)}-a_j| \norm{ e_j}  \to 0 \\
            & \Rightarrow \text{ strong convergence}
\end{flalign*}
\end{frame}

%----------------------------------
\begin{frame}\frametitle{Equivalence of Norms}

\begin{theorem}
 Let $\text{dim}(\mathbb{X})=k$ and let $\norm{\cdot} $ and $\norm{\cdot}_0$ be two different norms on $\mathbb{X}$ then $\exists a,b$ such that 
\[ 
a\norm{x}_0  \leq \norm{x} \leq b\norm{x}_0 
\]
\end{theorem}
\begin{proof}
 (in book page 96)
\end{proof}
 
{\bf Implication:}  For convergence proofs, it doesn't matter which norm you use, therefore, use the one that simplifies the proof.
\end{frame}


\end{document}