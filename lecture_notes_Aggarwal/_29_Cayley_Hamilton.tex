\documentclass{beamer}
\usepackage{amsmath,amsbsy,amsopn,amstext,amsfonts,amssymb}
\usepackage{isomath}
\usepackage{ulem}
%\linespread{1.6}  % double spaces lines
\usepackage{graphicx}
\usepackage{subfigure}
\usepackage{color}
\usepackage{optidef}  % define optimization problems
\usepackage{multicol}  % multiple columns
\usepackage{listings} % for python code
\usepackage{mathrsfs}

\usepackage{polynom}
\newcommand{\adj}{\mathrm{adj}}
\newcommand{\constrainedmin}[3]{
		\begin{mini*}|s|
		{#2}{#1}{}{}
		\addConstraint{#3}
		\end{mini*}
}

\newcommand{\rwbcomment}[1]{{\color{blue}RWB:#1}}
\newcommand{\defeq}{\stackrel{\triangle}{=}}
\newcommand{\abs}[1]{\left|#1\right|}
\newcommand{\norm}[1]{\left\|#1\right\|}
\newcommand{\iprod}[1]{\left<#1\right>}
\newcommand{\ellbf}{\boldsymbol{\ell}}
\newcommand{\nubf}{\boldsymbol{\nu}}
\newcommand{\mubf}{\boldsymbol{\mu}}
\newcommand{\abf}{\mathbf{a}}
\newcommand{\bbf}{\mathbf{b}}
\newcommand{\cbf}{\mathbf{c}}
\newcommand{\dbf}{\mathbf{d}}
\newcommand{\ebf}{\mathbf{e}}
\newcommand{\fbf}{\mathbf{f}}
\newcommand{\gbf}{\mathbf{g}}
\newcommand{\hbf}{\mathbf{h}}
\newcommand{\ibf}{\mathbf{i}}
\newcommand{\jbf}{\mathbf{j}}
\newcommand{\kbf}{\mathbf{k}}
\newcommand{\lbf}{\mathbf{l}}
\newcommand{\mbf}{\mathbf{m}}
\newcommand{\nbf}{\mathbf{n}}
\newcommand{\obf}{\mathbf{o}}
\newcommand{\pbf}{\mathbf{p}}
\newcommand{\qbf}{\mathbf{q}}
\newcommand{\rbf}{\mathbf{r}}
\newcommand{\sbf}{\mathbf{s}}
\newcommand{\tbf}{\mathbf{t}}
\newcommand{\ubf}{\mathbf{u}}
\newcommand{\vbf}{\mathbf{v}}
\newcommand{\wbf}{\mathbf{w}}
\newcommand{\xbf}{\mathbf{x}}
\newcommand{\ybf}{\mathbf{y}}
\newcommand{\zbf}{\mathbf{z}}
\newcommand{\Jbf}{\mathbf{J}}
\newcommand{\Acal}{\mathcal{A}}
\newcommand{\Bcal}{\mathcal{B}}
\newcommand{\Lcal}{\mathcal{L}}
\newcommand{\Ncal}{\mathcal{N}}
\newcommand{\Rcal}{\mathcal{R}}
\definecolor{darkolivegreen}{rgb}{0.33, 0.42, 0.18}

\makeatletter
\newenvironment<>{proofstart}[1][\proofname]{%
    \par
    \def\insertproofname{#1\@addpunct{.}}%
    \usebeamertemplate{proof begin}#2}
  {\usebeamertemplate{proof end}}
\newenvironment<>{proofcont}{%
  \setbeamertemplate{proof begin}{\begin{block}{}}
    \par
    \usebeamertemplate{proof begin}}
  {\usebeamertemplate{proof end}}
\newenvironment<>{proofend}{%
    \par
    \pushQED{\qed}
    \setbeamertemplate{proof begin}{\begin{block}{}}
    \usebeamertemplate{proof begin}}
  {\popQED\usebeamertemplate{proof end}}
\makeatother

\title{ECEn 671: Mathematics of Signals and Systems}
\author{Randal W. Beard}
\institute{Brigham Young University}
\date{\today}

\begin{document}

%-------------------------------
\begin{frame}
	\titlepage
\end{frame}

%%%%%%%%%%%%%%%%%%%%%%%%%%%%%%%%%%%%%%%%%%%%%%%%%%%%%%%%%%%%%%%%%
\section{Cayley-Hamilton Theorem}
\frame{\sectionpage}


%----------------------------------
\begin{frame}\frametitle{Functions of Matrices}
	\begin{lemma}
	A square matrix can always be put in Jordan form.	
	\end{lemma}
	This implies that we can always write
	\[ 
		A = SJS^{-1} 
	\]
	
	This implies that 
	\begin{align*}
		A^k &= \underbrace{AA\cdots A}_{k\text{ times}} \\
		    &= SJS^{-1}SJS^{-1}\cdots SJS^{-1} \\
		    &= SJ^kS^{-1} 
	\end{align*}
	This is particularly simple if $J = \Lambda$ since 
	\[ 
		A^k = S\Lambda^kS^{-1} 
		\text{ where } 
		\Lambda^k 
			= \begin{pmatrix}
	    		\lambda_1^k & & 0\\
	    		& \ddots \\
	    		0 & & \lambda_n^k
	  		  \end{pmatrix} 
	\]	
\end{frame}

%----------------------------------
\begin{frame}\frametitle{Functions of Matrices, cont.}
	For square matrices we can define analytic functions of matrices.
	Analytic functions are functions that can be expanded as a Taylor
	seires, e.g.
	\begin{align*}
		e^x &= 1 + x + \frac{x^2}{2!} + \frac{x^3}{3!} + \cdots \\
		\cos(x) &= 1 - \frac{x^2}{2!} + \frac{x^4}{4!} + \frac{x^6}{6!} + \cdots \\
		\sin(x) &= x - \frac{x^3}{3!} + \frac{x^5}{5!} - \frac{x^7}{7!} + \cdots 
	\end{align*}
	The corresponding Matrix function is defined in terms of its Taylor
	series, e.g.,
	\begin{align*}
		e^A &= I + A + \frac{A^2}{2!} + \frac{A^3}{3!} + \cdots\\
		\cos(A) &= I - \frac{A^2}{2!} + \frac{A^4}{4!} - \frac{A^6}{6!} + \cdots\\
		\sin(A) &= A - \frac{A^3}{3!} + \frac{A^5}{5!} - \frac{A^7}{7!} + \cdots
	\end{align*}

\end{frame}

%----------------------------------
\begin{frame}\frametitle{Cayley-Hamilton Theorem}
	Computing infinite series of matrices is a pain.  Fortunately we have
	the following theorem:
	
	\begin{theorem}[Cayley-Hamilton Theorem]
		Every matrix satisfies its own characteristic polynomial, i.e.
		\[ 
			\chi_A(A) = 0.
		\]	
	\end{theorem}
\end{frame}

%----------------------------------
\begin{frame}\frametitle{Cayley-Hamilton Theorem, proof}
	The proof holds for all $A$ but we will only prove the case when
	$q_i=m_i$ for each $\lambda_i$.  In this case $A=S\Lambda S^{-1}$.
	Note that for each eigenvalue $\chi_A(\lambda_i) = 0 $ since
	$\chi_A(\lambda_i) = \det(\lambda_i I - A)$
	
	Let \[
			\chi_A(\lambda) = \lambda^n + a_{n-1}\lambda^{n-1} + \cdots + a_1\lambda + a_0 
		\]
	then
	\begin{align*}
		\chi_A(A) &= A^n + a_{n-1}A^{n-1} + \cdots + a_1A + a_0I \\
				  &= S\Lambda^nS^{-1} + a_{n-1}S\Lambda^{n-1}S^{-1} + \cdots + a_1S\Lambda S^{-1} + a_0SS^{-1}\\
				  &= S(\Lambda^n + a_{n-1}\Lambda^{n-1} + \cdots + a_1\Lambda + a_0I)S^{-1}
	\end{align*}
	Note that the matrix 
	\[
		\Lambda^n + a_{n-1}\Lambda^{n-1} + \cdots + a_1\Lambda + a_0I
	\]
	is diagonal with each element on the diagonal equal to $\lambda_i^n +
	    a_{n-1}\lambda_i^{n-1} + \cdots + a_1\lambda_i + a_0=0$.
	  
	Therefore 
	\[
		\Lambda^n + a_{n-1}\Lambda^{n-1} + \cdots + a_1\Lambda + a_0I = 0.
	\]	
\end{frame}

%----------------------------------
\begin{frame}\frametitle{Cayley-Hamilton Theorem, implications}
	{\footnotesize
		Recall polynomial division:
		\begin{align*}
			& \frac{f(x)}{q(x)} = a(x) + \frac{r(x)}{q(x)} \\
			\implies & 
				\underbrace{f(x)}_{\text{degree $m$}} 
					=
					\underbrace{a(x)}_{\text{degree $(m-n)$}}
					\overbrace{q(x)}^{\text{degree $n$}} 
					+ \underbrace{r(x)}_{\text{degree $< n$}}
		\end{align*}
		Application to infinite series like $e^x$ gives
		\[ 
			\underbrace{e^x}_{\text{degree $\infty$}} 
				= \underbrace{a(x)}_{\text{degree $\infty$}}
				  \overbrace{\chi_A(x)}^{\text{degree $n$}} 
				  + \underbrace{r(x)}_{\text{degree $<n$}}
		\]
	}
	Since $\chi_A(A) = 0$,
	\begin{align*}
	 	e^A &= \underbrace{r(A)}_{\text{degree $< n$}}
			&= \cdots b_{n-1}A^{n-1} + \cdots + b_1A + b_0I
	\end{align*}
	Since $e^{\lambda_i} = r(\lambda_i)$ the coefficients can be found
	from
	\[ 
		e^{\lambda_i} = \cdots b_{n-1}\lambda^{n-1} + \cdots + b_1\lambda + b_0.
	\]	
\end{frame}

%----------------------------------
\begin{frame}\frametitle{Cayley-Hamilton Theorem, example}
	Find $e^A$ where 
	$A = \begin{pmatrix}
	    	0 & 1\\
	    	-2 & -2
	  	  \end{pmatrix}$.  
	{\footnotesize
	\[ \det(\lambda I - A ) = \lambda^2 + 2\lambda + 2 \Rightarrow
	\lambda_{1,2} = -1 \pm j \]
	\[ \Rightarrow e^A = b_1A + b_0I = \begin{pmatrix}
	    0 & b_1\\
	    -2b_1 & -2b_1
	  \end{pmatrix}+\begin{pmatrix}
	    b_0 & 0\\
	    0 & b_0
	  \end{pmatrix} = \begin{pmatrix}
	    b_0 & b_1\\
	    -2b_1 & -2b_1-b_0
	  \end{pmatrix}\]
	}
	where $b_0$ and $b_1$ satisfy
	\[e^{-1+j} = b_1(-1+j) + b_0\]
	\[e^{-1-j} = b_1(-1-j)+b_0 \]
	\[ \Rightarrow \begin{pmatrix}
	    b_0\\b_1
	  \end{pmatrix}
	= \begin{pmatrix}
	    1 & -1+j\\
	    1 & -1-j
	  \end{pmatrix}^{-1}\begin{pmatrix}
	    e^{-1+j}\\e^{-1-j}
	  \end{pmatrix}=\begin{pmatrix}
	    0.5083\\
	    0.3096
	  \end{pmatrix}
	\]
	\[\Rightarrow e^{A} = \begin{pmatrix}
	    0.5083 & 0.3096\\
	    -0.6191 & -0.1108
	  \end{pmatrix}
	 \]
\end{frame}



\end{document}