\documentclass{beamer}
\usepackage{amsmath,amsbsy,amsopn,amstext,amsfonts,amssymb}
\usepackage{isomath}
\usepackage{ulem}
%\linespread{1.6}  % double spaces lines
\usepackage{graphicx}
\usepackage{subfigure}
\usepackage{color}
\usepackage{optidef}  % define optimization problems
\usepackage{multicol}  % multiple columns
\usepackage{listings} % for python code
\usepackage{mathrsfs}

\usepackage{polynom}
\newcommand{\adj}{\mathrm{adj}}
\newcommand{\constrainedmin}[3]{
		\begin{mini*}|s|
		{#2}{#1}{}{}
		\addConstraint{#3}
		\end{mini*}
}

\newcommand{\rwbcomment}[1]{{\color{blue}RWB:#1}}
\newcommand{\defeq}{\stackrel{\triangle}{=}}
\newcommand{\abs}[1]{\left|#1\right|}
\newcommand{\norm}[1]{\left\|#1\right\|}
\newcommand{\iprod}[1]{\left<#1\right>}
\newcommand{\ellbf}{\boldsymbol{\ell}}
\newcommand{\nubf}{\boldsymbol{\nu}}
\newcommand{\mubf}{\boldsymbol{\mu}}
\newcommand{\abf}{\mathbf{a}}
\newcommand{\bbf}{\mathbf{b}}
\newcommand{\cbf}{\mathbf{c}}
\newcommand{\dbf}{\mathbf{d}}
\newcommand{\ebf}{\mathbf{e}}
\newcommand{\fbf}{\mathbf{f}}
\newcommand{\gbf}{\mathbf{g}}
\newcommand{\hbf}{\mathbf{h}}
\newcommand{\ibf}{\mathbf{i}}
\newcommand{\jbf}{\mathbf{j}}
\newcommand{\kbf}{\mathbf{k}}
\newcommand{\lbf}{\mathbf{l}}
\newcommand{\mbf}{\mathbf{m}}
\newcommand{\nbf}{\mathbf{n}}
\newcommand{\obf}{\mathbf{o}}
\newcommand{\pbf}{\mathbf{p}}
\newcommand{\qbf}{\mathbf{q}}
\newcommand{\rbf}{\mathbf{r}}
\newcommand{\sbf}{\mathbf{s}}
\newcommand{\tbf}{\mathbf{t}}
\newcommand{\ubf}{\mathbf{u}}
\newcommand{\vbf}{\mathbf{v}}
\newcommand{\wbf}{\mathbf{w}}
\newcommand{\xbf}{\mathbf{x}}
\newcommand{\ybf}{\mathbf{y}}
\newcommand{\zbf}{\mathbf{z}}
\newcommand{\Jbf}{\mathbf{J}}
\newcommand{\Acal}{\mathcal{A}}
\newcommand{\Bcal}{\mathcal{B}}
\newcommand{\Lcal}{\mathcal{L}}
\newcommand{\Ncal}{\mathcal{N}}
\newcommand{\Rcal}{\mathcal{R}}
\definecolor{darkolivegreen}{rgb}{0.33, 0.42, 0.18}

\makeatletter
\newenvironment<>{proofstart}[1][\proofname]{%
    \par
    \def\insertproofname{#1\@addpunct{.}}%
    \usebeamertemplate{proof begin}#2}
  {\usebeamertemplate{proof end}}
\newenvironment<>{proofcont}{%
  \setbeamertemplate{proof begin}{\begin{block}{}}
    \par
    \usebeamertemplate{proof begin}}
  {\usebeamertemplate{proof end}}
\newenvironment<>{proofend}{%
    \par
    \pushQED{\qed}
    \setbeamertemplate{proof begin}{\begin{block}{}}
    \usebeamertemplate{proof begin}}
  {\popQED\usebeamertemplate{proof end}}
\makeatother

\title{ECEn 671: Mathematics of Signals and Systems}
\author{Randal W. Beard}
\institute{Brigham Young University}
\date{\today}

\begin{document}

%-------------------------------
\begin{frame}
	\titlepage
\end{frame}


%%%%%%%%%%%%%%%%%%%%%%%%%%%%%%%%%%%%%%%%%%%%%%%%%%%%%%%%%%%%%%%%%
\section{SVD and Numerically Sensitive Problems}
\frame{\sectionpage}

%----------------------------------
\begin{frame}\frametitle{Numerically Sensitive Problems}
	Suppose that we would like to solve
	\[ 
		Ax = b  
	\]
	where $A \in \mathbb{R}^{n\times n}$ and $rank(A) = n$ but the condition number $\mathcal{K}(A)$ is large.  Let $A= U\Sigma V^H$, then 
	\begin{align*}
		 A^{-1} 
		 	&= V\Sigma^{-1} U^H\\
			&= \sum_{j=1}^{n}\frac{\vbf_j \ubf_j^H}{\sigma_j}
	\end{align*}
	
	so the solution to $Ax = b$ is
	\[ 
		x = A^{-1}b 
		  = \sum_{j=1}^{n}\frac{\vbf_j \ubf_j^H}{\sigma_j}.
	\]	
\end{frame}

%----------------------------------
\begin{frame}\frametitle{Numerically Sensitive Problems}
	Recall that 
	$\mathcal{K}(A) = \norm{A}\norm{A^{-1}}$ 
	where 
	$\norm{A} = \sigma_{\max}(A)$ 
	and 
	$\norm{A^{-1} } = \displaystyle\frac{1}{\min_{\norm{x}}\norm{A}} = \frac{1}{\sigma_{\min}(A)}$. 
		Therefore
	\[
		\mathcal{K}(A) = \frac{\sigma_{\max}(A)}{\sigma_{\min}(A)}
	\]
	
	\vfill
	
	Therefore a large $\mathcal{K}(A)$ implies there is significant difference between the largest and smallest singular values.	
\end{frame}

%----------------------------------
\begin{frame}\frametitle{Numerically Sensitive Problems}
	For example $\sigma_{\min}(A)$ may be very small, therefore given 
	\[ 
		x = \sum_{j=1}^n \frac{\vbf_j\ubf_j^H}{\sigma_j}b 
	\]
	x is very sensitive to small change in $b$ due to the terms in the sum that have very small singular values.	
	
	\vfill
	
	{\color{blue}Solution:}  
	Zero out small singular values to get the approximate solution
	\[ 
		Ax = \begin{pmatrix}
	  			U_1 & U_2  
			 \end{pmatrix}
			 \begin{pmatrix}
	    		\Sigma_1 & 0\\
	    		0 & \Sigma_2
	  		 \end{pmatrix}
	  		 \begin{pmatrix}
	    		V_1^H\\
	    		V_2^H
	  		 \end{pmatrix} x 
	  	\approx \begin{pmatrix}
	    			U_1 & U_2
	  			\end{pmatrix}
	  			\begin{pmatrix}
	    			\Sigma_1 & 0\\
	    			0 & 0
	  			\end{pmatrix}
	  			\begin{pmatrix}
	    			V_1^H\\
	    			V_2^H
	  			\end{pmatrix} x
	\]
	so
	\[ 
		x = V_1\Sigma_1^{-1}U_1^Hb 
	\]
	is an approximate solution that is numerically stable.
\end{frame}

%----------------------------------
\begin{frame}\frametitle{Numerically Sensitive Problems}
	\begin{itemize}
		\item Moon Example 7.4.1 shows that if $\sigma_j$-small then the vector $\ubf_j \in \mathbb{R}^m$ defines a sensitive direction for $b$.  i.e. if $b$ is almost parallel with $\ubf_j$ then $x = \frac{\vbf_j\ubf_j^H}{\sigma_j}b$ is clearly sensitive to small changes in $b$.  If $b$ is perpendicular to $\ubf_j$ then $\ubf_j^Hb=0$ and we are ok.
		\item If $A$ comes from noisy data (almost always) then $A$ will usually be full rank, even if the original data that produced $A$ would have resulted in a lower rank $A$ if it wasn't corrupted by noise.
		\item But the nonzero singular values added by noise will usually be small.
		\item Therefore, an effective way to reduce the rank of $A$ to get rid of the effect of noise is to zero the ``small'' singular values.
	\end{itemize}
\end{frame}




\end{document}