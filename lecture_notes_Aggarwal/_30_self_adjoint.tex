\documentclass{beamer}
\usepackage{amsmath,amsbsy,amsopn,amstext,amsfonts,amssymb}
\usepackage{isomath}
\usepackage{ulem}
%\linespread{1.6}  % double spaces lines
\usepackage{graphicx}
\usepackage{subfigure}
\usepackage{color}
\usepackage{optidef}  % define optimization problems
\usepackage{multicol}  % multiple columns
\usepackage{listings} % for python code
\usepackage{mathrsfs}

\usepackage{polynom}
\newcommand{\adj}{\mathrm{adj}}
\newcommand{\constrainedmin}[3]{
		\begin{mini*}|s|
		{#2}{#1}{}{}
		\addConstraint{#3}
		\end{mini*}
}

\newcommand{\rwbcomment}[1]{{\color{blue}RWB:#1}}
\newcommand{\defeq}{\stackrel{\triangle}{=}}
\newcommand{\abs}[1]{\left|#1\right|}
\newcommand{\norm}[1]{\left\|#1\right\|}
\newcommand{\iprod}[1]{\left<#1\right>}
\newcommand{\ellbf}{\boldsymbol{\ell}}
\newcommand{\nubf}{\boldsymbol{\nu}}
\newcommand{\mubf}{\boldsymbol{\mu}}
\newcommand{\abf}{\mathbf{a}}
\newcommand{\bbf}{\mathbf{b}}
\newcommand{\cbf}{\mathbf{c}}
\newcommand{\dbf}{\mathbf{d}}
\newcommand{\ebf}{\mathbf{e}}
\newcommand{\fbf}{\mathbf{f}}
\newcommand{\gbf}{\mathbf{g}}
\newcommand{\hbf}{\mathbf{h}}
\newcommand{\ibf}{\mathbf{i}}
\newcommand{\jbf}{\mathbf{j}}
\newcommand{\kbf}{\mathbf{k}}
\newcommand{\lbf}{\mathbf{l}}
\newcommand{\mbf}{\mathbf{m}}
\newcommand{\nbf}{\mathbf{n}}
\newcommand{\obf}{\mathbf{o}}
\newcommand{\pbf}{\mathbf{p}}
\newcommand{\qbf}{\mathbf{q}}
\newcommand{\rbf}{\mathbf{r}}
\newcommand{\sbf}{\mathbf{s}}
\newcommand{\tbf}{\mathbf{t}}
\newcommand{\ubf}{\mathbf{u}}
\newcommand{\vbf}{\mathbf{v}}
\newcommand{\wbf}{\mathbf{w}}
\newcommand{\xbf}{\mathbf{x}}
\newcommand{\ybf}{\mathbf{y}}
\newcommand{\zbf}{\mathbf{z}}
\newcommand{\Jbf}{\mathbf{J}}
\newcommand{\Acal}{\mathcal{A}}
\newcommand{\Bcal}{\mathcal{B}}
\newcommand{\Lcal}{\mathcal{L}}
\newcommand{\Ncal}{\mathcal{N}}
\newcommand{\Rcal}{\mathcal{R}}
\definecolor{darkolivegreen}{rgb}{0.33, 0.42, 0.18}

\makeatletter
\newenvironment<>{proofstart}[1][\proofname]{%
    \par
    \def\insertproofname{#1\@addpunct{.}}%
    \usebeamertemplate{proof begin}#2}
  {\usebeamertemplate{proof end}}
\newenvironment<>{proofcont}{%
  \setbeamertemplate{proof begin}{\begin{block}{}}
    \par
    \usebeamertemplate{proof begin}}
  {\usebeamertemplate{proof end}}
\newenvironment<>{proofend}{%
    \par
    \pushQED{\qed}
    \setbeamertemplate{proof begin}{\begin{block}{}}
    \usebeamertemplate{proof begin}}
  {\popQED\usebeamertemplate{proof end}}
\makeatother

\title{ECEn 671: Mathematics of Signals and Systems}
\author{Randal W. Beard}
\institute{Brigham Young University}
\date{\today}

\begin{document}

%-------------------------------
\begin{frame}
	\titlepage
\end{frame}




%%%%%%%%%%%%%%%%%%%%%%%%%%%%%%%%%%%%%%%%%%%%%%%%%%%%%%%%%%%%%%%%%
\section{Self Adjoint Matrices}
\frame{\sectionpage}

%----------------------------------
\begin{frame}\frametitle{Self Adjoint Matrices}
	\begin{definition}
	A matrix $A\in\mathbb{C}^{n\times n}$ is said to be \underline{self adjoint} (also called \underline{Hermitian}) if 
	\[
		A = A^H.
	\]	
	\end{definition}
	
	\begin{lemma}[Moon 6.2]
 		If $A = A^H$ then the eigenvalues of $A$ are real.
	\end{lemma}
	\begin{proof}
		Let $(\lambda,x)$ be a right eigen-pair, then
		\[ 
			Ax = \lambda x, \text{ and } x^HA^H = \bar{\lambda}x^H.
		\]
		Therefore 
		\begin{align*}
			& x^HAx = \lambda x^Hx, \text{ and } x^HA^Hx = \bar{\lambda}x^Hx \\
			\implies & \lambda x^Hx = \bar{\lambda}x^Hx  
			\implies  \bar{\lambda} = \lambda, \\
			\implies & \lambda \text{ is real. }	
		\end{align*}
	\end{proof}
\end{frame}

%----------------------------------
\begin{frame}\frametitle{Self Adjoint Matrices}
	\begin{lemma}[Moon 6.3]
		If $A=A^H$ and the eigenvalues are distinct, then the eigenvectors are orthogonal.
	\end{lemma}
	
	\begin{proof}
	Let $(\lambda_1,x_1)$ and $(\lambda_2,x_2)$ be distinct eigenpairs, i.e. $\lambda_1 \neq \lambda_2$, then
	\begin{align*}
		& x_2^HAx_1 = \lambda_1x_2^Hx_1 \\
		\text{and~} & x_2^HA^Hx_1 = \lambda_2x_2^Hx_1 \
	\end{align*}
	Therefore $(\lambda_1-\lambda_2)x_2^Hx_1 = 0$.
	Because $\lambda_1 \neq \lambda_2$ we must have that
	\[
		x_2^Hx_1 = 0
	\]
	which implies that $x_1$ and $x_2$ are orthogonal.
	\end{proof}
	{\color{blue}
		Note the eigenvectors can always be chosen to be orthonormal.
	}
\end{frame}

%----------------------------------
\begin{frame}\frametitle{Self Adjoint Matrices}

	\begin{theorem}[Moon Theorem 6.2 (Special Theorem)]
		If $A\in\mathbb{C}^{n\times n}$ is Hermitian,
		then $q_i = m_i$ for each eigenvalue $\lambda_i$.
	\end{theorem}
	
	\begin{corollary}
		If $A=A^H$, then $\exists$ a unitary $U$ and real diagonal $\Lambda$ such that
		\[
			A = U \Lambda U^H.
		\]
	\end{corollary}
	
\end{frame}

%----------------------------------
\begin{frame}\frametitle{Eigenvalues and Rank}
	
	\begin{lemma}[Moon Lemma 6.5]
		Let $A \in \mathbb{C}^{m\times m}$ be of rank $r < m$.  Then at least $m- r$ of the eigenvalues of $A$ are equal to zero	
	\end{lemma}
\end{frame}





\end{document}