\documentclass{beamer}
\usepackage{amsmath,amsbsy,amsopn,amstext,amsfonts,amssymb}
\usepackage{isomath}
\usepackage{ulem}
%\linespread{1.6}  % double spaces lines
\usepackage{graphicx}
\usepackage{subfigure}
\usepackage{color}
\usepackage{optidef}  % define optimization problems
\usepackage{multicol}  % multiple columns
\usepackage{listings} % for python code
\usepackage{mathrsfs}

\usepackage{polynom}
\newcommand{\adj}{\mathrm{adj}}
\newcommand{\constrainedmin}[3]{
		\begin{mini*}|s|
		{#2}{#1}{}{}
		\addConstraint{#3}
		\end{mini*}
}

\newcommand{\rwbcomment}[1]{{\color{blue}RWB:#1}}
\newcommand{\defeq}{\stackrel{\triangle}{=}}
\newcommand{\abs}[1]{\left|#1\right|}
\newcommand{\norm}[1]{\left\|#1\right\|}
\newcommand{\iprod}[1]{\left<#1\right>}
\newcommand{\ellbf}{\boldsymbol{\ell}}
\newcommand{\nubf}{\boldsymbol{\nu}}
\newcommand{\mubf}{\boldsymbol{\mu}}
\newcommand{\abf}{\mathbf{a}}
\newcommand{\bbf}{\mathbf{b}}
\newcommand{\cbf}{\mathbf{c}}
\newcommand{\dbf}{\mathbf{d}}
\newcommand{\ebf}{\mathbf{e}}
\newcommand{\fbf}{\mathbf{f}}
\newcommand{\gbf}{\mathbf{g}}
\newcommand{\hbf}{\mathbf{h}}
\newcommand{\ibf}{\mathbf{i}}
\newcommand{\jbf}{\mathbf{j}}
\newcommand{\kbf}{\mathbf{k}}
\newcommand{\lbf}{\mathbf{l}}
\newcommand{\mbf}{\mathbf{m}}
\newcommand{\nbf}{\mathbf{n}}
\newcommand{\obf}{\mathbf{o}}
\newcommand{\pbf}{\mathbf{p}}
\newcommand{\qbf}{\mathbf{q}}
\newcommand{\rbf}{\mathbf{r}}
\newcommand{\sbf}{\mathbf{s}}
\newcommand{\tbf}{\mathbf{t}}
\newcommand{\ubf}{\mathbf{u}}
\newcommand{\vbf}{\mathbf{v}}
\newcommand{\wbf}{\mathbf{w}}
\newcommand{\xbf}{\mathbf{x}}
\newcommand{\ybf}{\mathbf{y}}
\newcommand{\zbf}{\mathbf{z}}
\newcommand{\Jbf}{\mathbf{J}}
\newcommand{\Acal}{\mathcal{A}}
\newcommand{\Bcal}{\mathcal{B}}
\newcommand{\Lcal}{\mathcal{L}}
\newcommand{\Ncal}{\mathcal{N}}
\newcommand{\Rcal}{\mathcal{R}}
\definecolor{darkolivegreen}{rgb}{0.33, 0.42, 0.18}

\makeatletter
\newenvironment<>{proofstart}[1][\proofname]{%
    \par
    \def\insertproofname{#1\@addpunct{.}}%
    \usebeamertemplate{proof begin}#2}
  {\usebeamertemplate{proof end}}
\newenvironment<>{proofcont}{%
  \setbeamertemplate{proof begin}{\begin{block}{}}
    \par
    \usebeamertemplate{proof begin}}
  {\usebeamertemplate{proof end}}
\newenvironment<>{proofend}{%
    \par
    \pushQED{\qed}
    \setbeamertemplate{proof begin}{\begin{block}{}}
    \usebeamertemplate{proof begin}}
  {\popQED\usebeamertemplate{proof end}}
\makeatother

\title{ECEn 671: Mathematics of Signals and Systems}
\author{Randal W. Beard}
\institute{Brigham Young University}
\date{\today}

\begin{document}

%-------------------------------
\begin{frame}
	\titlepage
\end{frame}


%%%%%%%%%%%%%%%%%%%%%%%%%%%%%%%%%%%%%%%%%%%%%%%%%%%%%%%%%%%%%%%%%
\section{Gauss-Newton Optimization}
\frame{\sectionpage}

%----------------------------------
\begin{frame}\frametitle{Least Squares as a Gradient Descent Problem}
Consider the least squares problem
	\begin{mini*}|s|
		{x\in\mathbb{R}^n}{\norm{Ax-b}_2^2}{}{}
	\end{mini*}
where $A\in\mathbb{R}^{m\times n}$ is tall.  We know that the solution is
\[
x^\ast = (A^\top A)^{-1} A^\top b.
\]

Can we pose this as a gradient descent problem?
\end{frame}

%----------------------------------
\begin{frame}\frametitle{Least Squares as a Gradient Descent Problem}
	Define the \underline{residual} as
	\[
		\rbf(x) = \begin{pmatrix} r_1(x) \\ \vdots \\ r_m(x) \end{pmatrix} = Ax-b
	\]	
	and define the sum-of-squares error as
	\begin{align*}
	S(x) &= \frac{1}{2}\rbf^\top(x)\rbf(x) \\
		 &= \frac{1}{2}\sum_{j=1}^m r_j^2(x) \\
		 &= \frac{1}{2}(Ax-b)^\top (Ax-b) \\
		 &= \frac{1}{2}\norm{Ax-b}_2^2.
	\end{align*}
	The least squares problem is to find $x$ that minimizes $S(x)$.
\end{frame}

%----------------------------------
\begin{frame}\frametitle{Least Squares as a Gradient Descent Problem}
	The gradient of $S$ is given by
	\begin{align*}
		\frac{\partial S}{\partial x} 
			&= \frac{\partial \rbf}{\partial x}^\top(x) \rbf(x) \\
			&= A^\top (Ax-b) = A^\top A x - A^\top b.
	\end{align*}
	
	So the gradient descent algorithm gives
	\[
	x^{[k+1]} = x^{[k]} - \alpha \left(A^\top A x^{[k]} - A^\top b\right)
	\]
	
	In general, we might allow $\alpha>0$ to be a positive definite matrix $\mathscr{A}>0$:
	\[
	x^{[k+1]} = x^{[k]} - \mathscr{A} \left(A^\top A x^{[k]} - A^\top b\right).
	\]	
	
\end{frame}

%----------------------------------
\begin{frame}\frametitle{Least Squares as a Gradient Descent Problem}
	Selecting 
	\[
		\mathscr{A} = (A^\top A)^{-1}
	\]	
	gives
	\begin{align*}
	x^{[k+1]} 
		&= x^{[k]} - (A^\top A)^{-1} \left(A^\top A x^{[k]} - A^\top b\right) \\
		&= x^{[k]} - (A^\top A)^{-1} (A^\top A) x^{[k]} + (A^\top A)^{-1} A^\top b \\
		&= (A^\top A)^{-1} A^\top b,
	\end{align*}
	which is the optimal solution.  
	
	Noting that $A = \frac{\partial \rbf}{\partial x}$, we have shown that the iteration
	\[
	x^{[k+1]} = x^{[k]} - \left(\frac{\partial \rbf^\top}{\partial x}(x^{[k]}) \frac{\partial \rbf}{\partial x}(x^{[k]})\right)^{-1} \frac{\partial \rbf^\top}{\partial x}(x^{[k]}) \rbf(x^{[k]})
	\]
	converges to the optimal in {\em one} step when $\rbf(x) = Ax-b$.
\end{frame}

%----------------------------------
\begin{frame}\frametitle{Nonlinear Least Squares}
	Let $r_j(x)$, $j=1, \dots, m$ be a general set of residual function to be minimized.  In other words, suppose we wish to solve
	\begin{mini*}|s|
		{x\in\mathbb{R}^n}{\frac{1}{2}\rbf^\top(x) \rbf(x)}{}{}.
	\end{mini*}
	Let $\Jbf(x) \defeq \frac{\partial \rbf}{\partial x}(x)$.  Then the \underline{Gauss-Newton} (GN) iteration algorithm is given by
	\[
		x^{[k+1]} = x^{[k]} - \left(\Jbf^\top(x^{[k]}) \Jbf(x^{[k]})\right)^{-1} \Jbf^\top(x^{[k]}) \rbf(x^{[k]})
	\]
	
	We know that the GN method converges in one step for the linear least squares problem.
\end{frame}




\end{document}