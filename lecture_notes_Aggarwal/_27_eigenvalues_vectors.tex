\documentclass{beamer}
\usepackage{amsmath,amsbsy,amsopn,amstext,amsfonts,amssymb}
\usepackage{isomath}
\usepackage{ulem}
%\linespread{1.6}  % double spaces lines
\usepackage{graphicx}
\usepackage{subfigure}
\usepackage{color}
\usepackage{optidef}  % define optimization problems
\usepackage{multicol}  % multiple columns
\usepackage{listings} % for python code
\usepackage{mathrsfs}

\usepackage{polynom}
\newcommand{\adj}{\mathrm{adj}}
\newcommand{\constrainedmin}[3]{
		\begin{mini*}|s|
		{#2}{#1}{}{}
		\addConstraint{#3}
		\end{mini*}
}

\newcommand{\rwbcomment}[1]{{\color{blue}RWB:#1}}
\newcommand{\defeq}{\stackrel{\triangle}{=}}
\newcommand{\abs}[1]{\left|#1\right|}
\newcommand{\norm}[1]{\left\|#1\right\|}
\newcommand{\iprod}[1]{\left<#1\right>}
\newcommand{\ellbf}{\boldsymbol{\ell}}
\newcommand{\nubf}{\boldsymbol{\nu}}
\newcommand{\mubf}{\boldsymbol{\mu}}
\newcommand{\abf}{\mathbf{a}}
\newcommand{\bbf}{\mathbf{b}}
\newcommand{\cbf}{\mathbf{c}}
\newcommand{\dbf}{\mathbf{d}}
\newcommand{\ebf}{\mathbf{e}}
\newcommand{\fbf}{\mathbf{f}}
\newcommand{\gbf}{\mathbf{g}}
\newcommand{\hbf}{\mathbf{h}}
\newcommand{\ibf}{\mathbf{i}}
\newcommand{\jbf}{\mathbf{j}}
\newcommand{\kbf}{\mathbf{k}}
\newcommand{\lbf}{\mathbf{l}}
\newcommand{\mbf}{\mathbf{m}}
\newcommand{\nbf}{\mathbf{n}}
\newcommand{\obf}{\mathbf{o}}
\newcommand{\pbf}{\mathbf{p}}
\newcommand{\qbf}{\mathbf{q}}
\newcommand{\rbf}{\mathbf{r}}
\newcommand{\sbf}{\mathbf{s}}
\newcommand{\tbf}{\mathbf{t}}
\newcommand{\ubf}{\mathbf{u}}
\newcommand{\vbf}{\mathbf{v}}
\newcommand{\wbf}{\mathbf{w}}
\newcommand{\xbf}{\mathbf{x}}
\newcommand{\ybf}{\mathbf{y}}
\newcommand{\zbf}{\mathbf{z}}
\newcommand{\Jbf}{\mathbf{J}}
\newcommand{\Acal}{\mathcal{A}}
\newcommand{\Bcal}{\mathcal{B}}
\newcommand{\Lcal}{\mathcal{L}}
\newcommand{\Ncal}{\mathcal{N}}
\newcommand{\Rcal}{\mathcal{R}}
\definecolor{darkolivegreen}{rgb}{0.33, 0.42, 0.18}

\makeatletter
\newenvironment<>{proofstart}[1][\proofname]{%
    \par
    \def\insertproofname{#1\@addpunct{.}}%
    \usebeamertemplate{proof begin}#2}
  {\usebeamertemplate{proof end}}
\newenvironment<>{proofcont}{%
  \setbeamertemplate{proof begin}{\begin{block}{}}
    \par
    \usebeamertemplate{proof begin}}
  {\usebeamertemplate{proof end}}
\newenvironment<>{proofend}{%
    \par
    \pushQED{\qed}
    \setbeamertemplate{proof begin}{\begin{block}{}}
    \usebeamertemplate{proof begin}}
  {\popQED\usebeamertemplate{proof end}}
\makeatother

\title{ECEn 671: Mathematics of Signals and Systems}
\author{Randal W. Beard}
\institute{Brigham Young University}
\date{\today}

\begin{document}

%-------------------------------
\begin{frame}
	\titlepage
\end{frame}


%%%%%%%%%%%%%%%%%%%%%%%%%%%%%%%%%%%%%%%%%%%%%%%%%%%%%%%%%%%%%%%%%
\section{Eigenvalues and Eigenvectors}
\frame{\sectionpage}


%----------------------------------
\begin{frame}\frametitle{Eigenpair}
	Let $A\in\mathbb{C}^{n\times n}$.
	\begin{definition}
		\begin{itemize}
		  \item $(\lambda,x)$ is a \underline{right eigen-pair} if $Ax=\lambda x$ and $x \neq 0$.
		  \item $(\lambda,x)$ is a \underline{left eigen-pair} if $x^HA=\lambda x^H$ and $x\neq 0$.
		\end{itemize}		
	\end{definition}
	Note that $Ax = \lambda x$ can be written as
	\[ 
		(\lambda I - A)x = 0. 
	\]
	Therefore for $x$ to be an eigenvector (associated with $\lambda $) then $x\in\mathcal{N}(\lambda I-A)$, and
	\[
		x\neq 0 \Rightarrow \mathcal{N}(\lambda I-A) \neq \{0\} \Rightarrow det(\lambda I-A)=0 
	\]
	This formula can be used to find the eigenvalues and eigenvectors of a matrix.
\end{frame}

%----------------------------------
\begin{frame}\frametitle{Eigenpair: Example}
	Let $A = \begin{pmatrix}
	    0 & 1\\
	    -2 & -2
	  \end{pmatrix}$.  
	Find the eigenvalues and eigenvectors.
	
	\par\underline{Eigenvalues:}
		\[ 
		det(\lambda I - A) 
			= det\begin{pmatrix}
    				\lambda  & -1\\
   					 2 & \lambda +2
  				  \end{pmatrix} 
  			= \lambda^2 + 2\lambda  + 2 
  			= 0
  	\]
  	implies that
	\[ 
		\lambda  = -1 \pm \sqrt{1-2} 
		         = -1 \pm j 
	\]
	so that
	\[ 
		\lambda_1 = -1 + j 
		\qquad \qquad 
		\lambda_2 = -1 - j.
	\]
Which one is larger?  Note, there is no possible ordering among complex numbers.
	
\end{frame}

%----------------------------------
\begin{frame}\frametitle{Eigenpair: Example}
	\par\underline{Eigenvectors:}
	The eigenvectors can be found from the formula
	\(
		(\lambda I-A)x = 0.
	\)
	\[ 
		\lambda_1: 
			\begin{pmatrix}
	    		-1+j & -1\\
	    		2 & 1+j
	  		\end{pmatrix}
			\begin{pmatrix}
	    		x_{11}\\x_{12}
	  		\end{pmatrix}
			=
			\begin{pmatrix}
	    		0\\0
	  		\end{pmatrix}.
	\]
	Note that the rows are linearly dependent since
	\begin{align*} 
		& (-1 + j) \begin{pmatrix}
	    			-1 + j & -1
	  			  \end{pmatrix} 
	  	+ \begin{pmatrix}
	    	2 & 1+j
	  	  \end{pmatrix} \\
	  	&= 
			\begin{pmatrix}
	    		-2 & -(1+j)
	  		\end{pmatrix}
	  		+	\begin{pmatrix}
	    			2 & 1 + j
	  			\end{pmatrix} \\
	  	&= 0.
	\end{align*}
	Therefore, solving $(-1 + j)x_{11} - x_{12} = 0$ gives
	\[ 
		x_{12} = (-1+j)x_{11} 
	\]
	Let $x_{11} = 1$ then $x_{12} = -1 + j$.
	
	So 
	\[
		x_1 
			= \begin{pmatrix}
	    		1\\-1+j
	  		   \end{pmatrix}
	\]
	is an eigenvector.
\end{frame}

%----------------------------------
\begin{frame}\frametitle{Eigenpair: Example}
	\[ 
		\lambda_2: 
			\begin{pmatrix}
	    		-1-j & -1\\
	    		2 & 1-j
	  		\end{pmatrix}
			\begin{pmatrix}
	    		x_{21}\\
	    		x_{22}
	  		\end{pmatrix}
			= 
			\begin{pmatrix}
	    		0\\0
	  		\end{pmatrix}
	  \]
	Again the rows are linearly dependent so solve to get
	\(
		(-1-j)x_{21} = x_{22}
	\)
	Let $x_{21} = 1$ then $x_{22} = -1 - j$.
	
	So 
	\[
		x_2 = \begin{pmatrix}
	    		1 \\ -1 -j
	  		  \end{pmatrix}
	\]
	is an eigenvector.
\end{frame}

%----------------------------------
\begin{frame}\frametitle{Characteristic Polynomial}
	\begin{definition}
		The polynomial
		\[
			\chi_A(\lambda ) = det(\lambda I - A)
		\]
		is called the \underline{characteristic polynomial} of $A$.  The eigenvalues of $A$ are the roots of $\chi_A(\lambda )=0$.  The set of roots of $\chi_A(\lambda )=0$ is called the spectrum of $A$, denoted $\lambda (A)$.
	\end{definition}
\end{frame}

%----------------------------------
\begin{frame}\frametitle{Relationship between transfer function and state space models}

	Given a state space system:
	\begin{align*}
		\dot{x} &= Ax + Bu \\ 
		y &= Cx
	\end{align*}
	where $x\in \mathbb{R}^n$, $u \in \mathbb{R}^m$, $y \in \mathbb{R}^p$, what is the transfer function?
	
	Take the Laplace transform to get
	\begin{align*}
		sX(s) &= AX(s) + BU(s)\\
		Y(s) &= CX(s)
	\end{align*}
	
	From the first equation we get 
	\[ 
		X(s) = (sI - A)^{-1}BU(s) 
	\]
	
	From the second equation we get
	\[ 
		Y(s) = \underbrace{C(sI-A)^{-1}B}_{(p\times m) \text{ transfer matrix}}U(s) 
	\]
	
\end{frame}

%----------------------------------
\begin{frame}\frametitle{Relationship between transfer function and state space models}
	What are the poles of the system?
	\begin{align*}
		Y(s) &= C(sI-A)^{-1}BU(s) \\
		     &= \frac{C \adj(sI-A)B}{\det(sI-A)}U(s) 
	\end{align*}
	
	Therefore, the poles are when 
	\[
		det(sI-A) = 0,
	\]
	i.e. when $s$ is an eigenvalue of $A$.

	\vfill
	
	{\color{blue} 
		The poles of an LTI system and the eigenvalues of $A$ are equivalent!
	}
\end{frame}

%----------------------------------
\begin{frame}\frametitle{Generalized Eigenvalues}
	Eigenvalues and eigenvectors can be defined for more general operators than just matrices.
	
	\begin{example}
		Let $h(t)$ be the impulse response of an LTI system with Fourier transform $H(j\omega)$.
		\begin{center}
			\includegraphics[width=0.5\textwidth]{figures/chap6_linear_system}	
		\end{center}
		Recall that if $u(t)=e^{j\omega_0 t}$ then 
		\begin{align*}
		 y(t) &= |H(j\omega_0)|e^{j\left(\omega_0 t + \angle H(j\omega_0)\right)}\\
		&= |H(j\omega_0)|e^{j\angle H(j\omega_0)}e^{j\omega_0 t}
		\end{align*}
		i.e. if a sinusoid goes in then the output will be a sinusoid of the
		same frequency but different magnitude and phase.		
	\end{example}
\end{frame}

%----------------------------------
\begin{frame}\frametitle{Generalized Eigenvalues}
	\begin{lemma}
		Let $\Acal[u] = \displaystyle \int_0^\top  h(t-\tau)u(\tau)d\tau $ then
		\[ 
			(\lambda,x) = \left( H(j\omega)e^{j\angle H(j\omega)},e^{j\omega t} \right) 
		\]
		is an eigenpair of $\Acal$.
	\end{lemma}
	\begin{proof}
		\[ 
			\Acal[e^{j\omega t}] = \left( |H(j\omega)|e^{j\angle
		    H(j\omega)}\right) e^{j\omega t}.
		\]
		
		Note that for $\Acal$ there are an uncountable infinite number of eigenpairs.		
	\end{proof}
\end{frame}

%----------------------------------
\begin{frame}\frametitle{Geometric and Algebraic Multiplicity}
	\begin{definition}
		Factor the characteristic polynomial as follows:
		\[ 
			\chi_A(\lambda) = (\lambda-\lambda_1)^{m_1}(\lambda-\lambda_2)^{m_2}\cdots(\lambda-\lambda_p)^{m_p}
		\]
		$m_i$ is the \underline{algebraic multiplicity} of eigenvalue $\lambda_i$.	
	\end{definition}
	
	\begin{definition}
		The \underline{geometric multiplicity} of eigenvalue $\lambda_i$ is defined as 
		\[
			q_i = dim(\mathcal{N}(\lambda_iI-A)).
		\]
	\end{definition}
\end{frame}

%----------------------------------
\begin{frame}\frametitle{Geometric and Algebraic Multiplicity: Example}
	Let 
	$
		A = \begin{pmatrix}
	    		1 & 0 \\
	    		0 & 1
	  		\end{pmatrix}
	$ 
	then
	\[ 
		\chi_A(\lambda) 
			= \det(\lambda I - A) 
			= \det\begin{pmatrix}
	    			\lambda-1 & 0\\
					0 & \lambda-1
	  			   \end{pmatrix} 
	  		= (\lambda-1)^2. 
	\]
	Therefore, the algebraic multiplicity of $\lambda_1 = 1$ is $m_1 = 2$.
	
	What is the geometric multiplicity?
	\[ 
		q_1 = dim(\mathcal{N}(\lambda_1I-A)) 
			= dim\left(
				\mathcal{N}
					\begin{pmatrix}
	    				0 & 0\\
						0 & 0
	  				\end{pmatrix}
	  			\right) 
	  		= dim(\mathbb{R}^2) 
	  		= 2. 
	\]
	
	Note that the eigenvectors 
	$\begin{pmatrix}
	    \alpha \\ 0
	  \end{pmatrix}$ 
	and 
	$\begin{pmatrix}
	    0\\\beta
	  \end{pmatrix}$ 
	are linearly independent!
\end{frame}

%----------------------------------
\begin{frame}\frametitle{Geometric and Algebraic Multiplicity: Example}
	Let 
	$A = \begin{pmatrix}
	    	1 & 1\\
			0 & 1
	  	  \end{pmatrix}$
 	then 
	\[ 
	\chi_A(\lambda) 
		= \det\begin{pmatrix}
	    		\lambda-1 & -1\\
				0 & \lambda-1
	  		   \end{pmatrix} 
	 	= (\lambda-1)^2
	\]
	so the algebraic multiplicity of $\lambda_1 = 1$ is $m_1 = 2$.
	
	The geometric multiplicity is
	\begin{align*}
		q_1 
			&= dim(
				\mathcal{N}\{I-A\})
			= dim(\mathcal{N}
				\begin{pmatrix}
	    			0 & -1\\
					0 & 0
	  			\end{pmatrix}) \\
	  		&= dim(\{x\in\mathbb{R}^2 \mid x_2 = 0\}) 
			= 1 \neq m_1 
	\end{align*}
	
	What are the eigenvectors assocaited with $A$?
	\[ 
		(\lambda I-A)x 
			= \begin{pmatrix}
	    		0 & 1\\
				0 & 0
	  		  \end{pmatrix}
			  \begin{pmatrix}
	    		x_{11}\\
				x_{12}
	  		  \end{pmatrix} = 
			  \begin{pmatrix}
	    		x_{12}\\
				0
	  		  \end{pmatrix}
			= \begin{pmatrix}
	    		0 \\ 0
	  		  \end{pmatrix}
			\Rightarrow x_{12} = 0 
	\]
	so $\begin{pmatrix}
	    \alpha\\0
	  \end{pmatrix}$ are the eigenvectors associated with $\lambda_1$.  There are \underline{not} two linearly independent eigenvectors.
\end{frame}

%----------------------------------
\begin{frame}\frametitle{Linearly Independent Eigenvectors}
	In general we have,
	\begin{lemma}
		Let $A \in \mathbb{C}^{n\times n}$, then there are
		$n$-linearly independent eigenvectors if and only if 
		\[
		\text{algebraic multiplicity} = \text{geometric multiplicity}
		\]
		for each eigenvalue of $A$.
	\end{lemma}
\end{frame}

%----------------------------------
\begin{frame}\frametitle{Linearly Independent Eigenvectors: Proof}
	\begin{proofstart}
		First prove that if $\lambda_i \neq \lambda_j$ then
		\[ 
			\mathcal{N}(\lambda_iI-A) \cap
			\mathcal{N}(\lambda_jI-A)
			= \{0\}. 
		\]
		
		To prove the claim, suppose not, then 
		\[ 
			\exists x \neq 0 \text{ such that } x \in \mathcal{N}(\lambda_iI-A) \text{ and } x \in \mathcal{N}(\lambda_jI-A) 
		\]
		\begin{align*}
		\iff &\qquad Ax = \lambda_i x \text{ and } Ax = \lambda_j x \\
		\iff &\qquad \lambda_ix = \lambda_jx \\
		\iff &\qquad (\lambda_i-\lambda_j)x = 0 \\
		\iff &\qquad \lambda_i = \lambda_j
		\end{align*}
		which is a contradiction.	
	\end{proofstart}
\end{frame}

%----------------------------------
\begin{frame}\frametitle{Linearly Independent Eigenvectors: Proof}

	\begin{proofend}
	
		Note that the number of linearly independent eigenvectors associated with $\lambda_i$ is the geometric multiplicity $q_i$ since we can find $q_i$ linearly independent vectors that span $\mathcal{N}(\lambda_i-A)$.
	
		\vspace{0.5cm}
			
		The previous claim shows that if $x_i \in \mathcal{N}(\lambda_iI-A)$ then $x_i \notin \mathcal{N}(\lambda_jI-A)$ which implies that there are $\sum q_i$ linearly independent eigenvectors of $A$.  Since $\sum m_i = n$, the lemma follows.
		
		\vspace{0.5cm}
		
		Note that if the eigenvalues are all distinct then $m_i = 1$.  Also since $1 \leq q_i \leq m_i$, for each $i$, we must have that the algebraic multiplicity equals the geometric multiplicity.
		
	\end{proofend}
	
\end{frame}

%----------------------------------
\begin{frame}\frametitle{Linearly Independent Eigenvectors}
	Suppose that there are $n$-linearly independent eigenvectors (where some of the eigenvalues might be repeated), then we can write
	\begin{align*}
		&
		\begin{pmatrix} 
			Ax_1 & Ax_2 & \cdots & Ax_n
		\end{pmatrix}
		= \begin{pmatrix}
 			\lambda_1x_1 & \lambda_2x_2 & \cdots & \lambda_nx_n
 		  \end{pmatrix}	 \\
 		\iff &
		A 
		\underbrace{
			\begin{pmatrix} 
				x_1 & x_2 & \cdots & x_n
		  	\end{pmatrix}
		}_S
		= \underbrace{
			\begin{pmatrix} 
				x_1 & x_2 & \cdots & x_n
		  	\end{pmatrix}
		  }_S
		  \underbrace{
		  	\begin{pmatrix}
	    		\lambda_1 & 0 & \cdots & 0 \\
	    		& \ddots \\
	    		0 & 0 & \cdots & \lambda_n
	  	  	\end{pmatrix}
	  	  }_{\Lambda} \\
	  	\iff &
	  	AS = S\Lambda
	\end{align*}
	
	Since the eigenvectors are linearly independent, $S$ is invertible.  Therefore
	\begin{align*}
		& A = S\Lambda S^{-1}  \\
		\iff & \Lambda = S^{-1}AS 
	\end{align*}
	Therefore, we say that $S$ diagonalizes $A$.
\end{frame}

%----------------------------------
\begin{frame}\frametitle{Similarity Transformation}
	\begin{definition}
		Two matrices, $A$ and $B$ are said to be \underline{similar} if
		$\exists $ an invertible $T$ such that 
		\[ 
			A = TBT^{-1}.
		\]		
	\end{definition}

	\begin{lemma}
		Similar matrices have the same \underline{eigenvalues}.	
	\end{lemma}
	\begin{proof}
		Let $(\lambda,x)$ be an eigenpair of $A$, then
		\begin{align*}
			& Ax = \lambda x\\
			\iff & TBT^{-1}x = \lambda x\\
			\iff & BT^{-1}x = \lambda T^{-1}x\\
			\iff & By = \lambda y \text{ where } y = T^{-1}x\\
			\iff & (\lambda,y) \text{ is an eigenpair of }B\\
		\end{align*}		
	\end{proof}
	\end{frame}


\end{document}