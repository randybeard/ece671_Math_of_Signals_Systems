\documentclass{beamer}
\usepackage{amsmath,amsbsy,amsopn,amstext,amsfonts,amssymb}
\usepackage{isomath}
\usepackage{ulem}
%\linespread{1.6}  % double spaces lines
\usepackage{graphicx}
\usepackage{subfigure}
\usepackage{color}
\usepackage{optidef}  % define optimization problems
\usepackage{multicol}  % multiple columns
\usepackage{listings} % for python code
\usepackage{mathrsfs}

\usepackage{polynom}
\newcommand{\adj}{\mathrm{adj}}
\newcommand{\constrainedmin}[3]{
		\begin{mini*}|s|
		{#2}{#1}{}{}
		\addConstraint{#3}
		\end{mini*}
}

\newcommand{\rwbcomment}[1]{{\color{blue}RWB:#1}}
\newcommand{\defeq}{\stackrel{\triangle}{=}}
\newcommand{\abs}[1]{\left|#1\right|}
\newcommand{\norm}[1]{\left\|#1\right\|}
\newcommand{\iprod}[1]{\left<#1\right>}
\newcommand{\ellbf}{\boldsymbol{\ell}}
\newcommand{\nubf}{\boldsymbol{\nu}}
\newcommand{\mubf}{\boldsymbol{\mu}}
\newcommand{\abf}{\mathbf{a}}
\newcommand{\bbf}{\mathbf{b}}
\newcommand{\cbf}{\mathbf{c}}
\newcommand{\dbf}{\mathbf{d}}
\newcommand{\ebf}{\mathbf{e}}
\newcommand{\fbf}{\mathbf{f}}
\newcommand{\gbf}{\mathbf{g}}
\newcommand{\hbf}{\mathbf{h}}
\newcommand{\ibf}{\mathbf{i}}
\newcommand{\jbf}{\mathbf{j}}
\newcommand{\kbf}{\mathbf{k}}
\newcommand{\lbf}{\mathbf{l}}
\newcommand{\mbf}{\mathbf{m}}
\newcommand{\nbf}{\mathbf{n}}
\newcommand{\obf}{\mathbf{o}}
\newcommand{\pbf}{\mathbf{p}}
\newcommand{\qbf}{\mathbf{q}}
\newcommand{\rbf}{\mathbf{r}}
\newcommand{\sbf}{\mathbf{s}}
\newcommand{\tbf}{\mathbf{t}}
\newcommand{\ubf}{\mathbf{u}}
\newcommand{\vbf}{\mathbf{v}}
\newcommand{\wbf}{\mathbf{w}}
\newcommand{\xbf}{\mathbf{x}}
\newcommand{\ybf}{\mathbf{y}}
\newcommand{\zbf}{\mathbf{z}}
\newcommand{\Jbf}{\mathbf{J}}
\newcommand{\Acal}{\mathcal{A}}
\newcommand{\Bcal}{\mathcal{B}}
\newcommand{\Lcal}{\mathcal{L}}
\newcommand{\Ncal}{\mathcal{N}}
\newcommand{\Rcal}{\mathcal{R}}
\definecolor{darkolivegreen}{rgb}{0.33, 0.42, 0.18}

\makeatletter
\newenvironment<>{proofstart}[1][\proofname]{%
    \par
    \def\insertproofname{#1\@addpunct{.}}%
    \usebeamertemplate{proof begin}#2}
  {\usebeamertemplate{proof end}}
\newenvironment<>{proofcont}{%
  \setbeamertemplate{proof begin}{\begin{block}{}}
    \par
    \usebeamertemplate{proof begin}}
  {\usebeamertemplate{proof end}}
\newenvironment<>{proofend}{%
    \par
    \pushQED{\qed}
    \setbeamertemplate{proof begin}{\begin{block}{}}
    \usebeamertemplate{proof begin}}
  {\popQED\usebeamertemplate{proof end}}
\makeatother

\title{ECEn 671: Mathematics of Signals and Systems}
\author{Randal W. Beard}
\institute{Brigham Young University}
\date{\today}

\begin{document}

%-------------------------------
\begin{frame}
	\titlepage
\end{frame}


%%%%%%%%%%%%%%%%%%%%%%%%%%%%%%%%%%%%%%%%%%%%%%%%%%%%%%%%%%%%%%%%%%
\section{Generalized Fourier Series}
\frame{\sectionpage}

%----------------------------------
\begin{frame}\frametitle{Section 3.17: Generalized Fourier Series}

	Topic of interest: $L_2$ function approximation
	
	\vfill

	\begin{definition}[Complete Basis]
	An orthonormal set $\{p_i, i=1, \ldots, \infty\}$ in a Hilbert space
	$\mathbb{S}$ is a \underline{complete basis} or \underline{total basis} if
	$\forall x \in \mathbb{S}$
	\[ x = \sum_{i=1}^{\infty} \iprod{x,p_i }p_i \]
	\end{definition}
	
	\vfill

	Note that if $x = \sum_{i=1}^{\infty} c_ip_i$ and $\iprod{p_i,p_j } =
	\delta_{ij}$ then
	\[ \iprod{x,p_j } = \sum_{i=1}^{\infty}c_i\iprod{p_i,p_j }=c_j\]
	\[ \Rightarrow c_j = \iprod{x,p_j } \]

	
\end{frame}

%----------------------------------
\begin{frame}\frametitle{Generalized Fourier Series, cont.}
	Therefore we can write
	\[ 
	x = \sum_{i=1}^{\infty} \iprod{x,p_i }p_i.
	\]	
	
	\vfill
	
	Most common example:  standard Fourier basis
	\[ P_n(t) = \frac{1}{\sqrt{T}}e^{j \left(\frac{2\pi}{T}\right)n t} \]
	Any function $f \in L_2[0,T]$ can be written as
	\[ f(t) = \sum_{n=-\infty}^{\infty}c_n \frac{1}{T}
	e^{j\left(\frac{2\pi}{T}\right)nt} \]
	where the coefficients are given as
	\[ c_n = \iprod{f,\frac{1}{\sqrt{T}}e^{j\left(\frac{2\pi}{T}\right)nt} } \defeq
	\frac{1}{\sqrt{T}}\int_{0}^{T}f(t)e^{j\left(\frac{2\pi}{T}\right)nt}dt \]

\end{frame}

%----------------------------------
\begin{frame}\frametitle{Generalized Fourier Series, cont.}
	Actually it is common to place the $\frac{1}{\sqrt{T}}$'s together
	letting $f(t) = \sum_{n=-\infty}^{\infty} b_n e^{j\left(\frac{2\pi}{T}\right)nt}$ where
	\[ b_n = \iprod{f(t), \frac{1}{T}e^{j\left(\frac{2\pi}{T}\right)nt} } =
	\frac{1}{T}\int_0^{T}f(t) e^{-j\left(\frac{2\pi}{T}\right)nt}dt \]
	
	Generalized Fourier series hold for \underline{any} complete basis,
	i.e.
	\[ x = \sum_{j=1}^{\infty} \iprod{x,p_j }p_j \]
	
\end{frame}

%----------------------------------
\begin{frame}\frametitle{Generalized Fourier Series, cont.}
	There are two important relationship between a function and its
	Fourier transform.
	
	\begin{theorem}[Bessel's Inequality]
	Suppose $\{p_1,p_2,\ldots\}$ is orthonormal but not necessarily
	complete and let 
	\[ c = \{ \iprod{x,p_1 },\iprod{x,p_2 }, \ldots \} = \{c_1,c_2,\ldots
	\}\]
	then
	\[ \fbox{$\norm{c }_{\ellbf_2} \leq \norm{x }_{L_2}$} \]
	\end{theorem}

\end{frame}

%----------------------------------
\begin{frame}\frametitle{Proof:}

	\begin{align*}
	0 \leq \norm{x - \sum c_jp_j }_{L_2}^2 &= \iprod{x - \sum c_jp_j, x-
		\sum c_j p_j }_{L_2}\\
	&= \iprod{x,x }_{L_2} 
		- \sum \bar{c}_j\iprod{x,p_j }_{L_2}
	\\ \qquad & 
		- \sum c_j\bar{\iprod{x,p_j }}_{L_2} 
		+ \sum \sum c_j \bar{c}_k \iprod{p_j,p_k
	}_{L_2}\\
	&= \norm{x }_{L_2}^{2} - \sum \bar{c}_jc_j - \sum c_j\bar{c}_j + \sum
	c_j\bar{c}_j\\
	&= \norm{x }_{L_2}^2 - \sum_{j=1}^{\infty} |c_j|^2\\
	&=\norm{x }_{L_2}^2 - \norm{c }_{\ellbf_2}^2\\
	&\Rightarrow \norm{c }_{\ellbf_2}^2 \leq \norm{x }_{L_2}^2
	\end{align*}
\end{frame}

%----------------------------------
\begin{frame}\frametitle{Generalized Fourier Series, cont.}
	\begin{theorem}[Parseval's Equality]
	If $T = \{p_1,p_2,\ldots \}$ is complete then
	\[ \fbox{$\norm{x }_{L_2}^2 = \norm{c }_{\ellbf_2}^2$} \]
	\end{theorem}
	\begin{proof}
	If $T$ is complete then
	\[ \norm{x - \sum c_jp_j }^2 = 0 \]
	and the result follows from the proof of Bessel's inequality	.
	\end{proof}
\end{frame}

%----------------------------------
\begin{frame}\frametitle{Significance of Parseval's Equality}
	
	$\norm{x }_{L_2}^2 = \norm{c }_{\ellbf_2}^2$ says that the energy in a
	signal (i.e. $\norm{x }_{L_2}$) is equal to the energy in the
	Fourier coefficients (i.e. $\norm{c }_{\ellbf_2}^2$).
	
	\vfill
	
	This relationship between $x$ and its transform $c$ is written as 
	\[
	x \overset{\mathcal{F}}{\longleftrightarrow} c.
	\]
	
\end{frame}

%----------------------------------
\begin{frame}\frametitle{Significance of Parseval's Equality, cont.}
	\begin{lemma}[Moon Lemma 3.1]
	If $x \overset{\mathcal{F}}{\longleftrightarrow} c$ and $y \overset{\mathcal{F}}{\longleftrightarrow}
	b$ for the same complete basis $\{p_1,p_2,\ldots\}$ then
	\[ 
	\iprod{x,y }_{L_2} = \iprod{c,b }_{\ellbf_2}.
	\]		
	\end{lemma}
	\begin{proof}
	Let $x = \sum_{i=1}^{\infty} c_ip_i$, and  $y = \sum_{i=1}^{\infty}b_ip_i$
	then
	\begin{align*}
	\iprod{x,y }_{L_2} &=
	\sum_{i=1}^{\infty}\sum_{j=1}^{\infty}c_i\bar{b}_j\iprod{p_i,p_j }\\
	&= \sum_{i=1}^{\infty} c_i\bar{b}_i\\
	&= \iprod{c,b }_{\ellbf_2}
	\end{align*}		
	\end{proof}
\end{frame}




\end{document}